\documentclass[E:/GsjzTle/main/main.tex]{subfiles}
\begin{document}

\begin{enumerate}
\def\labelenumi{\arabic{enumi}.}
\item
  关于 \textbf{\#define int long long}

  \begin{quote}
  \begin{itemize}
  \item
    注意当题目某个类型为 \textbf{unsigned int} 时,如果没有去掉
    \textbf{\#define int longl ong},就会导致 \textbf{unsigned int}
    \(→\) \textbf{unsigned long long} 从而导致错误
  \item
    \textbf{\#define int long long} 后,\textbf{scanf} 和 \textbf{prinf}
    需要改成 \textbf{\%lld}
  \end{itemize}
  \end{quote}
\item
  题目中有\textbf{单个字母}读入和\textbf{数字}读入时,千万别用
  \(char - `0`\) 来代替数字,因为读入的数字可能\textbf{不是个位数}
\item
  注意是不是只建了\textbf{单向边}
\item
  题目是不是多组读入?
\item
  \(inf\) 是不是开得不够大或者太大了爆 \(long long\) ?
\item
  路径压缩的 \(find\) 和 按秩合并的 \(find\) 不同,别忘记修改!
\item
  检查除 \(void\) 类型的函数有没有\textbf{返回值}
\item
  一个数右移过多位再 \& 1 是无法得到准确值的 (int y = x
  \textgreater\textgreater{} 65 \& 1,此时无法确定 \(y\) 是等于 \(0\)
  还是 \(1\) )
\item
  \(reverse\) \(vector\) 注意判空 不然会 \(re\)
\item
  用位运算表示 \(2^n\) 注意加 \(1LL<<n\)
\item
  \(map.find\) 不会创建新元素,\(map[]\) 会,注意空间
\item
  涉及减法的时候取模时候要 \((+mod)\% mod\)
\end{enumerate}

\end{document}
