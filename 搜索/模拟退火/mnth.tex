\documentclass[E:/GsjzTle/main/main.tex]{subfiles}
\begin{document}
求费马点的例子,注意用真随机数。

\begin{lstlisting}
void solve()
{
	int i ;
	point ans , next ;
	point now ;
	ans.x = (p[1].x + p[2].x + p[3].x) / 3 ;
	ans.y = (p[1].y + p[2].y + p[3].y) / 3 ;
	double t = 100  ;
	double delta = 0.99 ;
	double dE ;
	double ans1 ;
	double inf = 1000000000 ;
	while(t > 1e-8)
	{
		ans1 = inf ;
		for(i = 0 ; i < 4 ; i ++)
		{
			next.x = ans.x + row[i] * t ;
			next.y = ans.y + col[i] * t ;
			if(cal(next) < ans1)
			{
				ans1 = cal(next) ;
				now = next ;
			}		   	 
		}
		dE = cal(now) - cal(ans) ;
		if(dE < 0.0 || exp(dE / t) > random1())
		{
			ans = now ;
		}
		t *= delta ;
	}
	printf("%.12f %.12f" , ans.x , ans.y) ;
}


\end{lstlisting}
\end{document}


