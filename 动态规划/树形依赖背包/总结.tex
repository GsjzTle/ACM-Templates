\documentclass[E:/GsjzTle/main/main.tex]{subfiles}

\begin{document}

\begin{itemize}
\item
  \textbf{问题模型}:\\
  有 \(N\)
  个产品,每个产品都有对应的体积和价值。对于其中某些产品,需要购买了它的父产品(最多只会有一个)才能购买它,问可以获得的最大价值?
\item
  \textbf{模型分析:}\\
  给有关系的产品连边,最后必然会生成一棵树或者一片森林\\
  有个小技巧:设置一个虚点 \(0\)
  成为所有树的根节点,那么森林就会化为一棵树了

  而对于树来说有个很重要的性质:

  \textbf{当回溯结束后,子树已经被遍历完并处理完了。所有的操作都是从叶子节点往上爬的!所以给
  \(dp\) 数组的赋初值的时候,可以考虑从叶子节点下手 !}

  \begin{enumerate}
  \def\labelenumi{\arabic{enumi}.}
  \item
    定义 \(dp_{u,i,j}\) 表示以 \(u\) 为根节点,从前 \(i\)
    个子节点中,选择了 \(j\) 个子节点的最大价值。
  \item
    像\(01\)背包一样压缩空间,得到:\(dp_{u,j}\) 表示 \(u\)
    为根节点,选择了 \(j\) 个子节点的最大价值。
  \item
    转移式为:\(dp_{u,j} = dp_{v,k} + dp_{u,j-k} + (u\rightarrow v\)
    的边权\()\)
  \item
    转移式中的 \(k < j\) (因为得保证 \(u\) 一定被选择了,不过有些题目
    \(k=j\) 也可,比如当 \(j\) 表示的是\textbf{叶子节点个数}的时候)
  \item
    若节点 \(u\) 有点权,则只需在初始化的时候令 \(dp_{u,1} = val_u\)
    即可
  \item
    若要设置虚点 \(0\)(将森林化为树),则最后答案为
    \(dp_{0,m+1}\)(多了一个节点 \(0\))
  \end{enumerate}
\item
  \textbf{树上背包也算是分组背包:}\\
  分组背包:若干组物品,每组中的物品互相冲突(就是说每组中最多只能选一件物品),求最大价值简单来说就是把每个节点看成一个背包啦,它的容量就是以这个节点为根的子树大小,组数就是连接的儿子个数。

  树上背包:每组都有很多选择,选一个,两个,三个,把这些选择当做组中的元素就好了,容易看出每组中只能选一个元素,比如你选择了选一个物品,就不可能同时选择选两个物品。
\end{itemize}

\begin{lstlisting}
const int N = 3e2 + 10;
struct Edge{
	int nex , to;
}edge[N << 1];
int head[N] , TOT;
void add_edge(int u , int v)
{
	edge[++ TOT].nex = head[u] ;
	edge[TOT].to = v;
	head[u] = TOT;
}
int n , m , dp[N][N] , s[N] , v[N] , sum = 0;
int dfs(int u , int far)
{
	int sum = 1;
	for(int i = head[u] ; i ; i = edge[i].nex)
	{
		int v = edge[i].to;
		if(v == far) continue ;
		sum += dfs(v , u);
		for(int j = sum ; j >= 1 ; j --)
		{
			for(int k = 0 ; k <= j - 1 ; k ++) 
			dp[u][j] = max(dp[u][j] , dp[u][j - k] + dp[v][k]); 
		}
	}
	return sum;
}
signed main()
{
	cin >> n >> m;
	for(int i = 1 ; i <= n ; i ++)
	{
		cin >> s[i] >> v[i];
		dp[i][1] = v[i];
		add_edge(i , s[i]);
		add_edge(s[i] , i);
	}
	dfs(0 , 0);
	cout << dp[0][m + 1] << '\n'; // 多了一个节点 0
	return 0;
}
\end{lstlisting}

\end{document}
