\documentclass[C:/Users/12748/Desktop/latex模板/main/main.tex]{subfiles}

\begin{document}

\begin{itemize}
\item
  设\(f[i][j]\) 表示在 \(a\) 的前 \(i\) 个数中取出,且以 \(b[j]\) 结尾的
  \(LCIS\) 长度。

  \begin{itemize}
  \item
    当 \(a[i] \ne b[j]\)时,我们不能选 \(a[i]\)(因为序列以 \(b[j]\)
    结尾,所以 \(b[j]\) 必须选),答案为\(f[i-1][j]\)
  \item
    当 \(a[i] = b[j]\) 时,我们枚举 \(b\) 之前可能的结尾,检查能不能把
    \(a[i]\)(即\(b[j]\)
    )加入,答案为\(\displaystyle \max_{a[i] > b[k]} \{f[i-1][k]+1\},\quad 0 \leq k < j\)
  \item
    最终答案为每一个可能为 \(b[j]\)
    结尾的\texttt{LIS}长度的最大值,即\(\max \{f[n][j]\}  ,1 \leq i \leq m\)
    复杂度 \(O(n^3)\)
  \end{itemize}
\item
  然后我们考虑 \(f[i][j]\) 可以由哪些状态转移过来,设这些状态为
  \(f[i][j]\) 的状态转移集合。

  我们发现这个集合包含 \(f[x][y], 0 \le x < i, 0 \le y \le j\)。\\
  考虑每次转移状态的时候这些状态的变化情况。\\
  我们就可以发现,已经在状态转移集合里的状态,在 \(i\) 或 \(j\)
  增加时不会离开集合。\\
  于是开个变量 \(val\),记录当前状态转移集合里的最大值,即
  \(\max \{f[i-1][k], 0 \le k < j\}\)\\
  在转移前面状态的同时维护后面状态的 \(val\) 即可。
\end{itemize}

\begin{lstlisting}
// O(N^2)
const int N = 5e3 + 10;
int n , m , a[N] , b[N];
int f[N][N] , pre[N][N] , ma;
void show(int i , int j)
{
	if(!i) return ;
	show(i - 1 , pre[i][j]);
	if(pre[i][j] != j) cout << b[j] << " "; 
	return;
}
pair<int , int> LCIS()
{
	rep(i , 1 , n)
	{
		int val = 0;
		rep(j , 1 , m)
		{
			if(a[i] == b[j])
			{
				f[i][j] = f[i - 1][val] + 1;
				pre[i][j] = val;
			}
			else
			{
				f[i][j] = f[i - 1][j];
				pre[i][j] = j;
			}
			if(b[j] < a[i] && f[i][j] > f[i][val]) val = j;
		}
	}
	int ma = 0;
	rep(j , 0 , m) if(f[n][j] > f[n][ma]) ma = j; 
	return make_pair(f[n][ma] , ma);
}
signed main()
{
	cin >> n;
	rep(i , 1 , n) cin >> a[i];
	cin >> m;
	rep(i , 1 , m) cin >> b[i];
	pair<int , int> res = LCIS();
	cout << res.first << '\n';
	show(n , res.second);
	return 0 ;
}
\end{lstlisting}

\begin{lstlisting}
// O(N^3)
const int N = 5e2 + 10;
int n , m , a[N] , b[N];
int f[N][N] , pre[N][N];
int ans , now;
vector<int>vec;
signed main()
{
	cin >> n;
	
	rep(i , 1 , n) cin >> a[i];
	
	cin >> m;
	
	rep(i , 1 , m) cin >> b[i];
	
	a[0] = b[0] = -1;
	
	rep(i , 1 , n) rep(j , 1 , m) 
	{
		if(a[i] == b[j])
		{
			rep(k , 0 , j - 1)
			{
				if(a[i] > b[k] && f[i][j] < f[i - 1][k] + 1)
				{
					f[i][j] = f[i - 1][k] + 1;
					pre[i][j] = k;
				}
			}
		}
		else
		{
			f[i][j] = f[i - 1][j];
			pre[i][j] = pre[i - 1][j];
		}
	}
	rep(j , 1 , m) if(ans < f[n][j])
	{
		ans = f[n][j];
		now = j;
	}
	while(now)
	{
		vec.push_back(b[now]);
		now = pre[n][now];
	}
	reverse(vec.begin() , vec.end());
	cout << ans << '\n';
	for(auto i : vec) cout << i << " ";
	cout << '\n';
	return 0;
}
\end{lstlisting}

\end{document}
