\documentclass[E:/GsjzTle/main/main.tex]{subfiles}

\begin{document}
	
	\begin{itemize}
		\item
		不管是求最长上升子序列还是最长不下降子序列最好都用
		\(upper\_bound\),用 \(lower\_bound\) 求不下降子序列会出错。
		\item
		求 \(lis\):
		
		\begin{enumerate}
			\def\labelenumi{\arabic{enumi}.}
			\item
			\(N^2\) 的 \(dp\)
			\item
			贪心 \(+\) 二分
			\item
			权值树状数组 \(or\) 权值线段树
		\end{enumerate}
		\item
		求 \(lis\) 方案数:\\
		定义 \(len\) 表示最大 \(lis\) 长度,\(f[i]\) 表示以 \(i\) 为结尾的
		\(lis\) 长度,\(cnt[i]\) 表示以 \(i\) 为结尾的 \(lis\) 个数,先预处理
		\(f[i]\) 和 \(len\)\\
		那么当 \(a[i] > a[j]~\&\&~f[i] = f[j] + 1\) 时,\(cnt[i] += cnt[j]\)\\
		答案为 \(\sum_{i=1}^{n}cnt[i][f[i] == len]\)\\
		如果要求方案数不重复的话,只要当 \(a[i] = a[j] \&\&f[i]=f[j]\)
		时,令\(cnt[i] = 0\)或\(cnt[j] = 0\) 即可
		\item
		求具体方案:\\
		和求方案数类似,先求出 \(f[i]\) 和 \(len\)
		
		\begin{enumerate}
			\def\labelenumi{\arabic{enumi}.}
			\item
			定义 \(pre[i]\) 表示 \(i\) 的前驱,那么当 \(f[i] = f[j] + 1\)
			时,\(pre[i] = j\)。
			\item
			从 \(f[now] = len\) 的位置开始,如果
			\(f[i] = f[now]-1\&\&a[i] < a[now]\) 则
			\(now = i\),不断重复此步骤直到 \(f[now] = 1\)。
		\end{enumerate}
		\item
		二维 \(lis\):\\
		先对第一维升序排序,再在排好序的序列上根据第二维求 \(lis\) 即可
	\end{itemize}
	
\end{document}
