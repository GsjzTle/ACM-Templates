\documentclass[E:/GsjzTle/main/main.tex]{subfiles}
\begin{document}

\begin{itemize}
\item
  一个正整数 \(n\)
  可以表示成若干个正整数之和,形如:\(n=n_1+n_2+…+n_k\),其中
  \(n_1≥n_2≥…≥n_k≥1\)
\item
  我们将这样的一种表示称为正整数 \(n\) 的一种划分(不考虑
  \(n_1,n_2,...,n_k\)) 顺序
\item
  现在给定一个正整数 \(n\),请你求出 \(n\) 共有多少种不同的划分方法
\end{itemize}

\begin{enumerate}
\def\labelenumi{\arabic{enumi}.}
\item
  可以认为 \(1 \sim n\) 每个数都有无限个,于是可以转换题意为:\\
  有 \(1\sim n\) 种物品,每种物品都有无限个,求背包容量为 \(n\)
  时能有多少种装包方案\\
  那么 \(f[n] += f[n - i],i∈[1 , n]\)
\item
  定义 \(f[n][m]\) 表示 \(n\) 划分成 \(m\) 个数的和的方案数\\
  ( \(n = n_1+n_2+...+n_m\),其中 \(n_i\) 可以为 \(0\))\\
  那么 :\\
  \(f[x][y] = f[x][y - 1] + f[x - i][y - 1] + f[x - 2\times i][y-1] + ...\)\\
  \(f[x][y] = f[x][y - 1] + f[x - i][y]\) \\
  又回到了完全背包。。。\\
  如果题目要求 划分的 \(m\) 个数不能为 \(0\)(不能为空),那么也好办:\\
  只要先让 \(m\) 个数都 \(+1\),再把剩下的 \(n-m\) 个数分给 \(m\)
  个就好,\(ans =f[n-m][m]\)
\item
  如果要求 \(n_1,n_2,...,n_k\) 互不相等\\
  那么不难发现先出来的数的数量一定不会超过 \(O(\sqrt{n})\),即
  \(m = min(m , \sqrt{n})\)\\
  复杂度为 \(O(n\sqrt{n})\)
\item
  如果 \(n_1,n_2,...,n_k\) 可以相等,但 \(n\)
  又很大,就需要用到生成函数\&五边形定理了\\
  复杂度为 \(O(n\sqrt{n})\)
\item
  如果 \(n_1,n_2,...,n_k\) 可以相等,但要求每个数出现的次数小于
  \(k\),那么还是得用上生成函数\&五边形定理,复杂度为 \(O(n\sqrt{n})\)
\end{enumerate}

\begin{itemize}
\item
  一:

\begin{lstlisting}
signed main()

{

	memset(f , 0 , sizeof(f));

	cin >> n;

	f[0] = 1;//一个都不选,那么就只有一种情况 

	for(int i = 1 ; i <= n ; i ++)

		for(int j = i ; j <= n ; j ++)

			f[j] = (f[j] + f[j - i]) % mod;

	cout << f[n] << '\n';

	return 0;

}
\end{lstlisting}
\item
  二:

\begin{lstlisting}
signed main()

{

	int n , k;

	cin >> n >> k;

	memset(f , 0 , sizeof(f));

	rep(i , 0 , n) f[0][i] = f[i][0] = f[i][1] = f[1][i] = 1; 

	rep(i , 2 , n)

	{

		rep(j , 2 , k)

		{

			if(i >= j) f[i][j] = f[i][j - 1] + f[i - j][j];

			else f[i][j] = f[i][j - 1];

		}

	}

	if(n - k < 0) cout << 0 << '\n';

	else cout << f[n - k][k] << '\n';

	return 0;

}
\end{lstlisting}
\item
  三略
\item
  四:

\begin{lstlisting}
const int N = 1e5 + 8;

const int mo = 1e9 + 7;

int dp[N];

signed main()

{

    int n = 1e5;

    dp[0] = 1;

    for (int i = 1; i <= n; ++i)

    {

        for (int j = 1, tmp = 1; i >= (3  * j * j - j) / 2 ; ++ j , tmp *= -1)

        {

            int x = (3 * j * j - j) / 2;

            int y = (3 * j * j + j) / 2;

            dp[i] = ((dp[i] + tmp * dp[i - x]) % mo + mo) % mo;

            if (i >= y) dp[i] = ((dp[i] + tmp * dp[i - y]) % mo + mo) % mo;

        }

    }

    int T = 1; cin >> T;

    while (T--)

    {

        int n ; cin >> n;

        cout << dp[n] << '\n';

    }

    return 0;

}
\end{lstlisting}
\item
  五:

\begin{lstlisting}
const int MOD = 1e9 + 7 ;

const int MAXN = 100001;

ll get_q(ll x)

{

	ll ans = 3 * x * x - x;

	return ans/2;

}

ll Q[MAXN] , P[MAXN];

void init()

{

	Q[0] = 0;

	for(int i = 1 ; i < MAXN ; i ++)

	{

		if(i & 1) Q[i] = get_q(i/2+1);

		else Q[i] = get_q(i/2*(-1));

	}

	P[0] = P[1] = 1;

	for(int i = 2 ; i < MAXN ; i ++)

	{

		for(int j = 1 ; ; j ++)

		{

			if(Q[j] > i) break;

			int t = j;

			if(t & 1) t = t / 2 + 1;

			else t = t / 2;

			if(t & 1) P[i] = P[i] + P[i - Q[j]];

			else P[i] = P[i] - P[i - Q[j]];

			if(P[i] > MOD) P[i] = P[i]%MOD;

			if(P[i] < 0) P[i] = P[i]+MOD;

		//	P[i] = (P[i] % MOD + MOD) % MOD;  这样写容易超时 

		}

	}

}

ll solve(int n , int k)

{

	ll ans = 0;

	for(int i = 0 ; ; i ++)

	{

		if(Q[i] * k > n) break;

		int t = i;

		if(t & 1) t = t / 2 + 1;

		else t = t / 2;

		if(t & 1) ans = ans - P[n - Q[i]*k];

		else ans = ans + P[n - Q[i]*k];

		if(ans > MOD) ans = ans % MOD;

		if(ans < 0) ans = ans + MOD;

	//	ans = (ans % MOD + MOD) % MOD;  // 这样写容易超时 

	}

	return ans;

}

signed main()

{

	init();

	int T = 1;

	cin >> T;

	while(T --)

	{

		int n , k;

		cin >> n >> k;

		cout << solve(n , k) << '\n';

	}

	return 0;

}
\end{lstlisting}
\end{itemize}

\end{document}
