\documentclass[E:/GsjzTle/main/main.tex]{subfiles}
\begin{document}

\begin{itemize}
\item
  \textbf{重剖}找 \(K\) 级祖先和\textbf{倍增}找 \(K\)
  级祖先的理论复杂度都是 \(log\)
\item
  但\textbf{重剖}的常数要比\textbf{倍增}来的小
\item
  不过对于森林来说\textbf{倍增}用起来会比较方便,因为\textbf{重剖}需要额外记录每棵树的
  \(dfn\_id\)
\item
  至于\textbf{长链剖分}虽然复杂度是 \(O(1)\)
  ,但是它得用到\textbf{倍增},以至于内存消耗大,所以不如就写个\textbf{重剖}来的
  \(easy\)
\end{itemize}

\end{document}
