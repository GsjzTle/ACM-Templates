\documentclass[E:/GsjzTle/main/main.tex]{subfiles}
\begin{document}
注意:特判边界,例如大小是$0$或$1$的行列式。


\begin{enumerate}
\item  如果出现负数高精度,尝试打表找规律。
	
\item  与容斥结合。画文氏图,分析容斥系数,确定加减。

\item  给出无向图,求这个图的生成树个数。

我们对这个图构造两个矩阵,分别是这个图的连通矩阵和度数矩阵。连通矩阵$S_1$的第$i$行第$j$列上的数字表示原无向图中编号为$i$和编号为$j$的两个点之间的边的条数。度数矩阵$S_2$只有斜对角线上有数字,即只有第$i$行第$i$列上有数字,表示编号为$i$的点的度数是多少。我们将两个矩阵相减,即$S_2−S_1$,我们记得到的矩阵为$T$,我们将矩阵$T$去掉任意一行和一列(一般情况去掉最后一行和最后一列的写法比较多)得到$T^{'}$,最后生成树的个数就是这个矩阵$T^{'}$的行列式。高斯消元求解行列式。


\item  给出有向图和其中的一个点,求以这个点为根的生成外向树个数。外向树是入度全为$1$,也就是每个节点指向儿子。

我们对这个图构造两个矩阵,分别是这个图的连通矩阵和度数矩阵。连通矩阵$S_1$的第$i$行第$j$列上的数字表示原无向图中编号为$i$和编号为$j$的两个点之间编号$i$的点指向编号为$j$的点的条数。度数矩阵$S_2$只有斜对角线上有数字,即只有第$i$行第$i$列上有数字,表示编号为$i$的点的入度是多少。我们将两个矩阵相减,即$S_2−S_1$,我们记得到的矩阵为$T$,我们将矩阵$T$去掉根所在行和根所在列得到$T^{'}$,最后生成树的个数就是这个矩阵$T^{'}$的行列式。高斯消元求解行列式。

\item  给出有向图和其中一个点,求以这个点为根的生成内向树个数。内向树是出度全为$1$,也就是每个节点指向父亲。

我们对这个图构造两个矩阵,分别是这个图的连通矩阵和度数矩阵。连通矩阵$S_1$的第$i$行第$j$列上的数字表示原无向图中编号为$i$和编号为$j$的两个点之间编号i的点指向编号为$j$的点的条数。度数矩阵$S_2$只有斜对角线上有数字,即只有第$i$行第$i$列上有数字,表示编号为$i$的点的出度是多少。我们将两个矩阵相减,即$S_2−S_1$,我们记得到的矩阵为$T$,我们将矩阵$T$去掉根所在行和根所在列得到$T^{'}$,最后生成树的个数就是这个矩阵$T^{'}$的行列式。高斯消元求解行列式。
\end{enumerate}

\end{document}


