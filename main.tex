\documentclass[10pt,a4paper]{article}
%\usepackage{zh_CN-Adobefonts_external}
\usepackage{xeCJK}
\usepackage{amsmath, amsthm}
\usepackage{listings,xcolor}
\usepackage{geometry} % 设置页边距
\usepackage{fontspec}
\usepackage{graphicx}
\usepackage[colorlinks]{hyperref}
\usepackage{setspace}
\usepackage{fancyhdr} % 自定义页眉页脚
\usepackage{subfiles}




\usepackage{unicode-math}
\usepackage{longtable,booktabs,array}
\makeatother
% Allow footnotes in longtable head/foot
\IfFileExists{footnotehyper.sty}{\usepackage{footnotehyper}}{\usepackage{footnote}}
\makesavenoteenv{longtable}





\setsansfont{Consolas} % 设置英文字体
\setmonofont[Mapping={}]{Consolas} % 英文引号之类的正常显示,相当于设置英文字体

\linespread{1.2}

\title{Template}
\author{GsjzTle}
\definecolor{dkgreen}{rgb}{0,0.6,0}
\definecolor{gray}{rgb}{0.5,0.5,0.5}
\definecolor{mauve}{rgb}{0.58,0,0.82}

\pagestyle{fancy}

\lhead{\CJKfamily{kai} Minnan Normal University} %以下分别为左中右的页眉和页脚
\chead{}

\rhead{\CJKfamily{kai} 第 \thepage 页}
\lfoot{} 
\cfoot{\thepage}
\rfoot{}
\renewcommand{\headrulewidth}{0.4pt} 
\renewcommand{\footrulewidth}{0.4pt}
%\geometry{left=2.5cm,right=3cm,top=2.5cm,bottom=2.5cm} % 页边距
\geometry{left=3.18cm,right=3.18cm,top=2.54cm,bottom=2.54cm}
\setlength{\columnsep}{30pt}

\makeatletter

\makeatother



\lstset{
	language    = c++,
	numbers     = left,
	numberstyle={                               % 设置行号格式
		\small
		\color{black}
		\fontspec{Consolas}
	},
	commentstyle = \color[RGB]{0,128,0}\bfseries, %代码注释的颜色
	keywordstyle={                              % 设置关键字格式
		\color[RGB]{40,40,255}
		\fontspec{Consolas Bold}
		\bfseries
	},
	stringstyle={                               % 设置字符串格式
		\color[RGB]{128,0,0}
		\fontspec{Consolas}
		\bfseries
	},
	basicstyle={                                % 设置代码格式
		\fontspec{Consolas}
		\small\ttfamily
	},
	emphstyle=\color[RGB]{112,64,160},          % 设置强调字格式
	breaklines=true,                            % 设置自动换行
	tabsize     = 4,
	frame       = single,%主题
	columns     = fullflexible,
	rulesepcolor = \color{red!20!green!20!blue!20}, %设置边框的颜色
	showstringspaces = false, %不显示代码字符串中间的空格标记
	escapeinside={\%*}{*)},
}

\begin{document}
	\title{赏金猎人}
	\author {GsjzTle}
	\maketitle
	\tableofcontents
	\newpage
	\section{模拟}
	\subsection{进制转换}
	\subfile{模拟/进制转换.tex}
	\section{搜索}
	\subsection{总结}
	\subfile{搜索/总结.tex}
	\subsection{三分}
	\subsubsection{整数、浮点数}
	\subfile{搜索/三分/sf.tex}
	\subsubsection{求极大值}
	\subfile{搜索/三分/求极大值.tex}
	\subsection{折半搜索}
	\subsubsection{总结}
	\subfile{搜索/折半搜索/zb.tex}
	\subsection{模拟退火}
	\subfile{搜索/模拟退火/mnth.tex}
	\section{贪心}
	\subsection{加工生产调度}
	\subfile{贪心/加工生产调度.tex}
	\subsection{反悔贪心}
	\subfile{贪心/反悔贪心.tex}
	\section{数论}
	\subsection{小经验}
	\subfile{数论/小经验.tex}
	\subsection{FWT}
	\subsubsection{序列中选 1,2,...,n 个数的最大异或和分别为多少}
	\subfile{数论/FWT/序列中选1,2,...,n个数的最大异或和分别为多少.tex}
	\subsection{Min25筛}
	\subsubsection{快速阶层取模}
	\subfile{数论/Min25筛/快速阶层取模.tex}
	\subsection{Pohlig-Hellman}
	\subsubsection{总结}
	\subfile{数论/Pohlig-Hellman/总结.tex}
	\subsubsection{Pohlig-Hellman}
	\subfile{数论/Pohlig-Hellman/Pohlig-Hellman.tex}
	\subsection{二次剩余}
	\subsubsection{总结}
	\subfile{数论/二次剩余/总结.tex}
	\subsubsection{2019牛客多校B}
	\subfile{数论/二次剩余/2019牛客多校B.tex}
	\subsubsection{一元二次方程判断是否有解}
	\subfile{数论/二次剩余/一元二次方程判断是否有解.tex}
	\subsubsection{求所有解}
	\subfile{数论/二次剩余/求所有解.tex}
	\subsection{反素数}
	\subsubsection{总结}
	\subfile{数论/反素数/总结.tex}
	\subsubsection{求不超过n的最小反素数}
	\subfile{数论/反素数/求不超过n的最小反素数.tex}
	\subsubsection{求约数个数为n最小反素数}
	\subfile{数论/反素数/求约数个数为n最小反素数.tex}
	\subsection{各种筛}
	\subsubsection{x的质因子个数}
	\subfile{数论/各种筛/x的质因子个数.tex}
	\subsubsection{矩阵的lcm}
	\subfile{数论/各种筛/矩阵的lcm.tex}
	\subsubsection{素数+欧拉函数+最小质因子}
	\subfile{数论/各种筛/素数+欧拉函数+最小质因子.tex}
	\subsection{康托展开}
	\subsubsection{康托展开}
	\subfile{数论/康托展开/康托展开.tex}
	\subsubsection{康托逆展开}
	\subfile{数论/康托展开/康托逆展开.tex}
	\subsection{扩展欧几里得}
	\subsubsection{exgcd}
	\subfile{数论/扩展欧几里得/exgcd.tex}
	\subsubsection{求同余方程}
	\subfile{数论/扩展欧几里得/求同余方程.tex}
	\subsection{离散对数}
	\subsubsection{BSGS}
	\subfile{数论/离散对数/BSGS.tex}
	\subsubsection{EX\_BSGS}
	\subfile{数论/离散对数/EX_BSGS.tex}
	\subsection{欧拉函数}
	\subsubsection{欧拉函数性质}
	\subfile{数论/欧拉函数/欧拉函数性质.tex}
	\subsubsection{sqrt(N)}
	\subfile{数论/欧拉函数/sqrt(N).tex}
	\subsubsection{欧拉筛}
	\subfile{数论/欧拉函数/欧拉筛.tex}
	\subsubsection{N\^(0.25)+1e18}
	\subfile{数论/欧拉函数/N^(0.25)+1e18.tex}
	\subsection{欧拉降幂}
	\subsubsection{费马小定理}
	\subfile{数论/欧拉降幂/费马小定理.tex}
	\subsubsection{欧拉定理}
	\subfile{数论/欧拉降幂/欧拉定理.tex}
	\subsubsection{扩展欧拉定理}
	\subfile{数论/欧拉降幂/扩展欧拉定理.tex}
	\subsection{排列组合}
	\subsubsection{总结}
	\subfile{数论/排列组合/总结.tex}
	\subsubsection{排列组合公式}
	\subfile{数论/排列组合/排列组合公式.tex}
	\subsubsection{暴力求法}
	\subfile{数论/排列组合/暴力求法.tex}
	\subsubsection{线性(N要小于p)}
	\subfile{数论/排列组合/线性(N要小于p).tex}
	\subsubsection{卢卡斯}
	\subfile{数论/排列组合/卢卡斯.tex}
	\subsubsection{扩展卢卡斯}
	\subfile{数论/排列组合/扩展卢卡斯.tex}
	\subsubsection{错排}
	\subfile{数论/排列组合/错排.tex}
	\subsection{裴蜀定理}
	\subsubsection{总结}
	\subfile{数论/裴蜀定理/总结.tex}

	\subsection{唯一分解定理+约数定理}
	\subsubsection{总结}
	\subfile{数论/唯一分解定理+约数定理/总结.tex}
	\subsubsection{约数之和}
	\subfile{数论/唯一分解定理+约数定理/约数之和.tex}
	\subsubsection{求二元倒数方程}
	\subfile{数论/唯一分解定理+约数定理/求二元倒数方程.tex}
	\subsubsection{具有N个不同因子的最小正整数}
	\subfile{数论/唯一分解定理+约数定理/具有N个不同因子的最小正整数.tex}
	\subsection{线性基}
	\subsubsection{总结}
	\subfile{数论/线性基/总结.tex}
	\subsubsection{验证存在性+最小值+最大值+第K小}
	\subfile{数论/线性基/验证存在性+最小值+最大值+第K小.tex}
	\subsubsection{最大异或和路径}
	\subfile{数论/线性基/最大异或和路径.tex}
	\subsubsection{区间询问异或和或上K的最大值}
	\subfile{数论/线性基/区间询问异或和或上K的最大值.tex}
	\subsection{整除分块}
	\subsubsection{总结}
	\subfile{数论/整除分块/总结.tex}
	\subsubsection{余数求和}
	\subfile{数论/整除分块/余数求和.tex}
	\subsubsection{有限小数对数}
	\subfile{数论/整除分块/有限小数对数.tex}
	\subsection{中国剩余定理}
	\subsubsection{总结}
	\subfile{数论/中国剩余定理/总结.tex}
	\subsubsection{中国剩余定理}
	\subfile{数论/中国剩余定理/中国剩余定理.tex}
	\subsubsection{扩展中国剩余定理}
	\subfile{数论/中国剩余定理/扩展中国剩余定理.tex}
	\section{图论}
	\subsection{小经验}
	\subfile{图论/小经验.tex}
	\subsection{K短路}
	\subsubsection{Astar}
	\subfile{图论/K短路/Astar.tex}
	\subsection{K级祖先}
	\subsubsection{总结}
	\subfile{图论/K级祖先/总结.tex}
	\subsubsection{倍增法}
	\subfile{图论/K级祖先/倍增法.tex}
	\subsubsection{重链剖分}
	\subfile{图论/K级祖先/重链剖分.tex}
	\subsubsection{长链剖分}
	\subfile{图论/K级祖先/长链剖分.tex}
	\subsection{LCA}
	\subsubsection{tarjan离线}
	\subfile{图论/LCA/tarjan离线.tex}
	\subsubsection{倍增法}
	\subfile{图论/LCA/倍增法.tex}
	\subsubsection{树剖法}
	\subfile{图论/LCA/树剖法.tex}
	\subsection{次小生成树}
	\subsubsection{kruskal+lca}
	\subfile{图论/次小生成树/kruskal+lca.tex}
	\subsection{带花树}
	\subsubsection{一般图最大匹配}
	\subfile{图论/带花树/一般图最大匹配.tex}
	\subsubsection{一般图最大权匹配}
	\subfile{图论/带花树/一般图最大权匹配.tex}
	\subsection{矩阵树定理}
	\subsubsection{总结}
	\subfile{图论/矩阵树定理/总结.tex}
	\subsubsection{无向图生成树个数}
	\subfile{图论/矩阵树定理/无向图生成树个数.tex}
	\subsubsection{有(无)向图生成树的权值和}
	\subfile{图论/矩阵树定理/有(无)向图生成树的权值和.tex}
	\subsection{判断完全图}
	\subsubsection{判断完全图}
	\subfile{图论/判断完全图/判断完全图.tex}
	\subsection{树的直径}
	\subsubsection{两次dfs}
	\subfile{图论/树的直径/两次dfs.tex}
	\subsection{最短路}
	\subsubsection{Dijkstra堆优化}
	\subfile{图论/最短路/Dijkstra堆优化.tex}
	\subsubsection{Dijkstra配对堆}
	\subfile{图论/最短路/Dijkstra配对堆.tex}
	\subsection{最小mex生成树}
	\subsubsection{最小mex生成树}
	\subfile{图论/最小mex生成树/最小mex生成树.tex}
	\section{动态规划}
	\subsection{总结}
	\subfile{动态规划/总结.tex}
	\subsection{01背包}
	\subsubsection{前k优解}
	\subfile{动态规划/01背包/前k优解.tex}
	\subsubsection{求最优解方案个数}
	\subfile{动态规划/01背包/求最优解方案个数.tex}
	\subsubsection{求恰好装满背包的最优解}
	\subfile{动态规划/01背包/求恰好装满背包的最优解.tex}
	\subsubsection{求具体方案}
	\subfile{动态规划/01背包/求具体方案.tex}
	\subsubsection{超大背包}
	\subfile{动态规划/01背包/超大背包.tex}
	\subsection{完全背包}
	\subsubsection{求最优解方案数}
	\subfile{动态规划/完全背包/求最优解方案数.tex}
	\subsection{多重背包}
	\subsubsection{总结}
	\subfile{动态规划/多重背包/总结.tex}
	\subsection{分组背包}
	\subsubsection{总结}
	\subfile{动态规划/分组背包/总结.tex}
	\subsection{树形依赖背包}
	\subsubsection{总结}
	\subfile{动态规划/树形依赖背包/总结.tex}
	\subsection{状压dp}
	\subsubsection{总结}
	\subfile{动态规划/状压dp/总结.tex}
	\subsection{区间dp}
	\subsection{数位dp}
	\subsection{经典问题}
	\subsubsection{最长上升子序列}
	\subfile{动态规划/经典问题/最长上升子序列.tex}
	\subsubsection{最长公共子序列}
	\subfile{动态规划/经典问题/最长公共子序列.tex}
	\subsubsection{最短不公共子串、子序列}
	\subfile{动态规划/经典问题/最短不公共子串、子序列.tex}
	\subsubsection{最长公共上升子序列}
	\subfile{动态规划/经典问题/最长公共上升子序列.tex}
	\subsubsection{整数划分}
	\subfile{动态规划/经典问题/整数划分.tex}
	\section{数据结构}
	\subsection{CDQ分治}
	\subsubsection{cdq}
	\subfile{数据结构/CDQ分治/cdq.tex}
	\subsubsection{数值删除、逆序对个数(排列)}
	\subfile{数据结构/CDQ分治/数值删除、逆序对个数(排列).tex}
	\subsection{splay}
	\subsubsection{插入x、删除x、查x的排名、查排名为x的数、x的前驱、x的后继}
	\subfile{数据结构/splay/1.tex}
	\subsubsection{区间插入、区间删除、区间覆盖、区间翻转、区间求和、最大子段和}
	\subfile{数据结构/splay/2.tex}
	\subsection{dsu on tree}
	\subsubsection{总结}
	\subfile{数据结构/dsuontree/总结.tex}
	\subsubsection{CF600E}
	\subfile{数据结构/dsuontree/CF600E.tex}
	\subsubsection{CF570D}
	\subfile{数据结构/dsuontree/CF570D.tex}
	\subsubsection{CF208E}
	\subfile{数据结构/dsuontree/CF208E.tex}
	\subsubsection{CF246E}
	\subfile{数据结构/dsuontree/CF246E.tex}
	\subsubsection{CF375D}
	\subfile{数据结构/dsuontree/CF375D.tex}
	\subsubsection{CF1009F}
	\subfile{数据结构/dsuontree/CF1009F.tex}
	\subsubsection{wannafly Day2 E}
	\subfile{数据结构/dsuontree/wannaflyDay2E阔力梯的树.tex}
	\subsubsection{ccpc2020长春站F题}
	\subfile{数据结构/dsuontree/ccpc2020长春站F题.tex}
	\subsubsection{洛谷P4149}
	\subfile{数据结构/dsuontree/洛谷P4149.tex}
	\subsubsection{牛客练习赛60E}
	\subfile{数据结构/dsuontree/牛客练习赛60E旗鼓相当的对手.tex}
	\subsubsection{牛客练习赛81D}
	\subfile{数据结构/dsuontree/牛客练习赛81D.tex}
	\subsubsection{CF741D}
	\subfile{数据结构/dsuontree/CF741D.tex}
	\subsection{单调栈}
	\subfile{数据结构/单调栈/单调栈.tex}
	\subsection{单调队列}
	\subsubsection{理想正方形}
	\subfile{数据结构/单调队列/理想正方形.tex}
	\subsection{第二分块}
	\subsubsection{将区间内x改为y、区间第k大}
	\subfile{数据结构/第二分块/将区间内x改为y、区间第k大.tex}
	\subsubsection{区间大于x的数减去x、区间内x出现的次数}
	\subfile{数据结构/第二分块/区间大于x的数减去x、区间内x出现的次数.tex}
	\subsection{树状数组}
	\subsubsection{二维树状数组}
	\subfile{数据结构/树状数组/二维树状数组.tex}
	\subsection{线段树}
	\subsubsection{总结}
	\subfile{数据结构/线段树/总结.tex}
	\subsubsection{01序列、区间覆盖为0、区间覆盖为1、区间取反、区间1的个数、区间最大连续1的个数}
	\subfile{数据结构/线段树/01序列、区间覆盖为0、区间覆盖为1、区间取反、区间1的个数、区间最大连续1的个数.tex}
	\subsubsection{单点修改、区间重排是否在值域上连续}
	\subfile{数据结构/线段树/单点修改、区间重排是否在值域上连续.tex}
	\subsubsection{单点修改、区间最大连续子段和}
	\subfile{数据结构/线段树/单点修改、区间最大连续子段和.tex}
	\subsubsection{动态区间中位数}
	\subfile{数据结构/线段树/动态区间中位数.tex}
	\subsubsection{区间+k、区间最大连续子段和}
	\subfile{数据结构/线段树/区间+k、区间最大连续子段和.tex}
	\subsubsection{区间+k、区间最大值、区间最小值、区间和}
	\subfile{数据结构/线段树/区间+k、区间最大值、区间最小值、区间和.tex}
	\subsubsection{区间gcd}
	\subfile{数据结构/线段树/区间gcd.tex}
	\subsubsection{区间hash}
	\subfile{数据结构/线段树/区间hash.tex}
	\subsubsection{区间乘法}
	\subfile{数据结构/线段树/区间乘法.tex}
	\subsubsection{区间覆盖+查询}
	\subfile{数据结构/线段树/区间覆盖+查询.tex}
	\subsubsection{区间异或x、区间求和}
	\subfile{数据结构/线段树/区间异或x、区间求和.tex}
	\subsubsection{区间与、区间或、区间最小值}
	\subfile{数据结构/线段树/区间与、区间或、区间最小值.tex}
	\subsubsection{树上区间+k、区间×k、区间覆盖、区间立方和}
	\subfile{数据结构/线段树/树上区间+k、区间×k、区间覆盖、区间立方和.tex}
	\subsubsection{矩阵+k、矩阵最大值}
	\subfile{数据结构/线段树/矩阵+k、矩阵最大值.tex}
	\subsection{吉司机线段树}
	\subsubsection{区间+k、区间覆盖、区间最大值、区间历史最大值}
	\subfile{数据结构/吉司机线段树/区间+k、区间覆盖、区间最大值、区间历史最大值.tex}
	\subsection{主席树}
	\subsubsection{撤销X个操作(可撤销先前撤销的操作)}
	\subfile{数据结构/主席树/撤销X个操作(可撤销先前撤销的操作).tex}
	\subsubsection{区间出现次数大于某值的最小数}
	\subfile{数据结构/主席树/区间出现次数大于某值的最小数.tex}
	\subsubsection{区间第K大}
	\subfile{数据结构/主席树/区间第K大.tex}
	\subsubsection{区间内不同数的个数}
	\subfile{数据结构/主席树/区间内不同数的个数.tex}
	\subsubsection{区间内只出现过一次的数}
	\subfile{数据结构/主席树/区间内只出现过一次的数.tex}
	\subsubsection{区间询问最小不能被表示的数}
	\subfile{数据结构/主席树/区间询问最小不能被表示的数.tex}
	\subsubsection{区间众数}
	\subfile{数据结构/主席树/区间众数.tex}
	\subsubsection{子树中距离其不超过K的最小点权}
	\subfile{数据结构/主席树/子树中距离其不超过K的最小点权.tex}
	\subsection{树套树}
	\subsubsection{单点修改、区间第k大}
	\subfile{数据结构/树套树/单点修改、区间第k大.tex}
	\subsubsection{数值删除、逆序对数(排列)}
	\subfile{数据结构/树套树/数值删除、逆序对数(排列).tex}
	\subsection{块状链表}
	\subfile{数据结构/块状链表/块状链表.tex}
	\subsection{普通并查集}
	\subsubsection{路径压缩}
	\subfile{数据结构/普通并查集/路径压缩.tex}
	\subsubsection{按秩合并}
	\subfile{数据结构/普通并查集/按秩合并.tex}
	\subsubsection{启发式合并+路径压缩}
	\subfile{数据结构/普通并查集/启发式合并+路径压缩.tex}
	\subsection{种类并查集}
	\subsection{可撤销并查集}
	\subsubsection{可撤销并查集}
	\subfile{数据结构/可撤销并查集/可撤销并查集.tex}
	\subsubsection{加边、删边、判二分图}
	\subfile{数据结构/可撤销并查集/加边、删边、判二分图.tex}
	\subsubsection{加边、删边、判联通}
	\subfile{数据结构/可撤销并查集/加边、删边、判联通.tex}
	\subsection{可持久化并查集}
	\subsubsection{总结}
	\subfile{数据结构/可持久化并查集/总结.tex}
	\subsubsection{可持久化并查集}
	\subfile{数据结构/可持久化并查集/可持久化并查集.tex}
	\subsection{带权并查集}
	\subsubsection{带权并查集}
	\subfile{数据结构/带权并查集/带权并查集.tex}
	\subsection{双指针}
	\subfile{数据结构/双指针/szz.tex}
	\section{字符串}
	\subsection{字典树}
	\subsubsection{指针版(释放内存)}
	\subfile{字符串/字典树/指针版(释放内存).tex}
	\subsubsection{字典树}
	\subfile{字符串/字典树/字典树.tex}
	\subsection{字符串hash}
	\subsubsection{总结}
	\subfile{字符串/字符串hash/总结.tex}
	\subsubsection{字符串hash}
	\subfile{字符串/字符串hash/字符串hash.tex}
	\subsubsection{二维hash、矩阵hash}
	\subfile{字符串/字符串hash/二维hash、矩阵hash.tex}
	\section{头脑风暴}
	\subsection{dp合集}
	\subsubsection{不等式约束(牛客)}
	\subfile{头脑风暴/dp合集/不等式约束(牛客).tex}
	\subsubsection{三元组的约束(AT)}
	\subfile{头脑风暴/dp合集/三元组的约束(AT).tex}
	\subsection{等差子序列}
	\subsubsection{总结}
	\subfile{头脑风暴/等差子序列/总结.tex}
	\subsection{回文问题}
	\subsubsection{寻找回文路径(AT)}
	\subfile{头脑风暴/回文问题/寻找回文路径(AT).tex}
	\subsection{排列问题}
	\subsubsection{字典序最小的(排列\&子序列)}
	\subfile{头脑风暴/排列问题/字典序最小的(排列&子序列).tex}
	\subsection{区间多次询问问题}
	\subsubsection{从区间中选出两个数使得异或值最大}
	\subfile{头脑风暴/区间多次询问问题/从区间中选出两个数使得异或值最大.tex}
	\subsection{数论只会gcd?}
	\subsubsection{求给定范围内gcd为素数的数对有多少?}
	\subfile{头脑风暴/数论只会gcd?/1.求给定范围内gcd为素数的数对有多少?.tex}
	\subsubsection{求给定范围内gcd为素数的数对有多少?}
	\subfile{头脑风暴/数论只会gcd?/2.求给定范围内gcd为素数的数对有多少?.tex}
	\subsubsection{单点乘法取模+整体gcd}
	{头脑风暴/数论只会gcd?/3.单点乘法取模+整体gcd.tex}
	\subsubsection{给定c,d,x求满足c×lcm(a,b) - d×gcd(a,b) = x 的 (a,b) 的对数}
	\subfile{头脑风暴/数论只会gcd?/4.给定c,d,x求满足c×lcm(a,b)-d×gcd(a,b)=x的(a,b)的对数.tex}
	\subsection{以子串为单位的倍增思想}
	\subsubsection{以子串为单位的倍增思想}
	\subfile{头脑风暴/以子串为单位的倍增思想/以子串为单位的倍增思想.tex}
	\subsection{最大字段和问题}
	\subsubsection{从序列中选出K个子串使得子串和最大}
	\subfile{头脑风暴/最大字段和问题/从序列中选出K个子串使得子串和最大.tex}
	\section{杂项}
	\subsection{对拍}
	\subsubsection{ac.cpp}
	\subfile{杂项/对拍/1.tex}
	\subsubsection{wa.cpp}
	\subfile{杂项/对拍/2.tex}
	\subsubsection{data.cpp}
	\subfile{杂项/对拍/make.tex}
	\subsubsection{duipai.cpp}
	\subfile{杂项/对拍/duipai.tex}
	\subsection{素数表}
	\subfile{杂项/素数表.tex}
	\subsection{小tricks}
	\subfile{杂项/小tricks.tex}
	\subsection{玄学优化}
	\subfile{杂项/玄学优化.tex}
	\subsection{头文件}
	\subfile{杂项/头文件.tex}
	\section{赛前看一看}
	\subsection{错误征集}
	\subfile{赛前看一看/错误征集.tex}
	\subsection{运行结果和自己预期的不一样?}
	\subfile{赛前看一看/运行结果和自己预期的不一样?.tex}
\end{document}