\documentclass[E:/GsjzTle/main/main.tex]{subfiles}
\begin{document}

\(c\times lcm(a,b) - d\times gcd(a , b) = x\)\\
\(c\times \dfrac{a \times b}{gcd(a,b)} - d\times gcd(a , b) = x\)\\
\(c \times a\times b - d\times gcd(a , b)^2 = x \times gcd(a,b)\)\\
\(c \times a\times b = x \times gcd(a,b) + d\times gcd(a , b)^2\)\\
\(c \times \dfrac{a}{gcd(a,b)}\times \dfrac{b}{gcd(a,b)} = \dfrac{x}{gcd(a,b)} + d\)\\
因为 \(c \times \dfrac{a}{gcd(a,b)}\times \dfrac{b}{gcd(a,b)}\)
为整数,\(d\) 为整数\\
所以 \(\dfrac{x}{gcd(a,b)}\) 为整数,所以 \(gcd(a,b)\) 为 \(x\) 的因子\\
于是 \(gcd(a,b)\) 我们可以在 \(\sqrt{x}\) 的复杂度内枚举\\
定义 \(A = \dfrac{a}{gcd(a,b)} , B = \dfrac{b}{gcd(a,b)}\)\\
那么有 \(gcd(A , B) =1\)
,\(A \times B = \dfrac{\dfrac{x}{gcd(a,b)} + d}{c}\)\\
定义 \(R =  {\dfrac{x}{gcd(a,b)} + d}\)\\
因为 \(A \times B\) 为整数,所以 \(R\) 要满足 \(R\bmod c = 0\)\\
于是问题就转换成 有多少对 \(A,B\) 满足
\(\begin{cases}\gcd \left( A,B\right) =1\\
A\times B=R\end{cases}\)\\
对于 \(A \times B= R\),只要将 \(R\) 的质因子分配给 \(A\) 或 \(B\) 即可
\\
对于 \(gcd(A , B) = 1\),只要将 \(R\) 的某个质因子全部分给 \(A\)
或者全部分给 \(B\) 即可\\
那么 \(R\) 的贡献就是 \(2^{p_{cnt}}\) ,其中 \(p_{cnt}\) 表示 \(R\)
的质因子个数

\begin{lstlisting}
const int M = 2e7 + 10;
int cnt[M];
signed main()
{
	int T , up = 2e7;
	for(int i = 2 ; i <= up ; i ++)
	{
		if(cnt[i]) continue ;
		for(int j = i ; j <= up ; j += i) cnt[j] ++ ;
	}
	cin >> T;
	while(T --)
	{
		int c , d , x , res = 0;	
		cin >> c >> d >> x;
		for(int i = 1 ; i * i <= x ; i ++)
		{
			if(x % i == 0)
			{
				int R = x / i + d;
				if(R % c == 0)
				{
					R /= c;
					res += 1ll << cnt[R];	
				} 	
				int j = x / i;			
				if(j == i) continue ;			
				R = x / j + d;			
				if(R % c == 0)
				{
					R /= c;			
					res += 1ll << cnt[R];
				}
			}
		}	
		cout << res << '\n';
	}
	return 0;
}
\end{lstlisting}

\end{document}
