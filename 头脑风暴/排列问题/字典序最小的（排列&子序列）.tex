\documentclass[E:/GsjzTle/main/main.tex]{subfiles}

\begin{document}
题意:
\begin{itemize}
\item
  给定一个长度为 \(N\) 的序列 \(A\) 和 \(K\)
\item
  满足 \(\forall i \in[1 , N],A[i] <= K\)
\item
  现要求选择一个长度为 \(K\) 的子序列
\item
  使得这个子序列是一个排列且字典序最小
\end{itemize}
思路:
\begin{quote}
我们按顺序从前往后在这 \(N\) 个数之中选 \(K\) 个数出来

一个自然的想法:

每次尽量选取字典序最小的数,让较大的数在答案序列中尽量靠后

我们可以用类似单调队列的思想去实现:

用 \(ans[]\) 存储答案,当枚举到第 \(i\) 个数 \(A[i]\) 的时候,

若 \(ans[]\) 之中没有 \(A[i]\) 时,将其与 \(ans[]\) 的最后一位 \(B\)
做比较

如果 \(A[i] < b\) 且 \(A[i]\) 之后还有那么删去 \(B\),接着比较,直到在
\(ans[]\) 中找到不满足条件的数为止

此时插入 \(A[i]\),枚举下一个数
\end{quote}

\begin{lstlisting}
rep(i , 1 , n) 
{
  if(vis[a[i]]) continue ;
  while(deq.size() && a[deq.back()] > a[i] && last[a[deq.back()]] > i) vis[a[deq.back()]] = 0 , deq.pop_back();   
  deq.pb(i);
  vis[a[i]] = 1;	
}
for(auto i : deq) cout << a[i] << " ";
\end{lstlisting}

\end{document}
