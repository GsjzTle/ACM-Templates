\documentclass[E:/GsjzTle/main/main.tex]{subfiles}

\begin{document}

\textbf{problem:}

\begin{quote}
给定 \(N\) 个点 \(M\) 条边,每条边都对应一个小写字母

问是否存在一条从 \(1\) 到 \(N\)
的路径,使得路径上的字母构成的字符串为回文串

\begin{itemize}
\item
  若存在则输出回文串的最短长度,若不存在则输出 \(-1\)
\end{itemize}
\end{quote}

\textbf{solve}

\begin{quote}
考虑双向 \(bfs + dp\) (以 \(dp\) 来思考会好理解许多)\\
从 \(1\) \textasciitilde{} \(n\) 的路径构成回文相当于:

\begin{itemize}
\item
  从 \(1\) \textasciitilde{} \(i\) 的路径和从 \(i\) \textasciitilde{}
  \(n\) 的路径构成回文
\end{itemize}

或

\begin{itemize}
\item
  从 \(1\) \textasciitilde{} \(i\) 的路径和从 \(j\) \textasciitilde{}
  \(n\) 的路径构成回文,其中 \(i、j\)相邻
\end{itemize}

于是可以定义 \(dp[i][j]\) 来判断从\(1\) 到 \(i\) 的路径和从 \(n\) 到
\(j\) 的路径是否构成回文

\begin{itemize}
\item
  \(dp[i][j] = 1\) 表示构成回文串
\item
  \(dp[i][j] = 0\) 表示不构成回文串
\end{itemize}

有点类似从两头一起出发往中间靠的区间 \(dp?\) \\
起初一头在 \(1\) 号点,另一头在 \(N\) 号点\\
起初 \(1\)\textasciitilde{}\(1\)
之间没有路径,\(N\)\textasciitilde{}\(N\) 之间也没有路径\\
所以初始化 \(dp[1][n] = 1\),并把 \((1,n)\) 打包存入队列进行双向
\(bfs\)\\
在 \(bfs\) 的过程中:

\begin{itemize}
\item
  当 \(i = j\) 或 \(i、j\) 相邻时更新答案
\item
  当 \(i\) 的相邻边和 \(j\) 的相邻边相同时,定义 \(i\) 的相邻节点为
  \(ii\),\(j\) 的相邻节点为 \(jj\),那么可由 \(dp[i][j] = 1\) 推出
  \(dp[ii][jj] = 1\),然后把 \((ii,jj)\) 打包存入队列继续 \(bfs\)
\end{itemize}
\end{quote}

\begin{lstlisting}
	const int N = 1e3 + 10;
	int n , m , mp[N][N] , dp[N][N]; 
	vector<int>G[N][30]; 
	signed main()
	{
		cin >> n >> m ;
		for(int i = 1 ; i <= m ; i ++)
		{
			int u , v;
			char ch;
			cin >> u >> v >> ch;
			G[u][ch - 'a'].push_back(v);
			G[v][ch - 'a'].push_back(u);
			mp[u][v] = mp[v][u] = 1;
		}
		int res = 1e9;
		queue<pair<pair<int , int> , int>>que;
		que.push(make_pair(make_pair(1 , n) , 0));
		dp[1][n] = 1;
		while(!que.empty())
		{
			pair<pair<int , int> , int>q = que.front();
			que.pop();
			int x = q.fi.fi , y = q.fi.se , z = q.se;
			if(z > res) break ;
			if(x == y) res = min(res , z);
			if(mp[x][y]) res = min(res , z + 1);
			for(int k = 0 ; k <= 25 ; k ++)
			{
				for(auto i : G[x][k])
				{
					for(auto j : G[y][k])
					{
						if(dp[i][j]) continue ;
						dp[i][j] = 1;
						que.push(make_pair(make_pair(i , j) , z + 2));
					}
				}
			}
		}
		if(res == 1e9) res = -1;
		cout << res << '\n';
		return 0;
	}
\end{lstlisting}

\end{document}
