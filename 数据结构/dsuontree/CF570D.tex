\documentclass[E:/GsjzTle/main/main.tex]{subfiles}

\begin{document}

\textbf{题目大意}

\begin{quote}
给定一个以\(1\)为根的n个节点的树,每个点上有一个字母\((a-z)\)

每个点的深度定义为该节点到1号节点路径上的点数.

每次询问 \(a,b\), 查询以 \(a\) 为根的子树内深度为 \(b\)
的节点上的字母重新排列之后是否能构成回文串.
\end{quote}

\textbf{解题思路}

\begin{quote}
重排之后能否构成回文串,即判断出现次数为奇数的字符个数是否小于等于1

因为要多次询问以某个点为根深度为 \(b\) 的节点上的字母

所以可以对每个节点开个 \(vector\) ,再把和它有关的询问保存在\(vector\)里

然后在 \(dsu\) \(on\) \(tree\) 的过程进行到该节点时遍历这个节点的
\(vector\) (相关询问)

再对询问做判断→得到答案→保存答案

最后将保存的答案按照读入的顺序排个序输出即可
\end{quote}

\begin{lstlisting}
const int N = 5e5 + 10;
struct Edge
{
	int nex , to;
} edge[N << 2];
int head[N] , TOT;
void add_edge(int u , int v)
{
	edge[++ TOT].nex = head[u];
	edge[TOT].to = v;
	head[u] = TOT;
}
int cnt[N][26] , hson[N] , sz[N] , HH;
int n , m , col[N] , vis[N];
vector<pair<int ,int>>Q[N] , ans;
void dfs(int u , int far)
{
	for(int i = head[u] ; i ; i = edge[i].nex)
	{
		int v = edge[i].to;
		if(v == far) continue ;
		dfs(v , u);
		sz[u] += sz[v];
		if(sz[v] > sz[hson[u]]) hson[u] = v;
	}
	sz[u] ++;
}
void calc(int u , int far , int val , int dep)
{
	cnt[dep][col[u]] += val;
	for(int i = head[u] ; i ; i = edge[i].nex)
	{
		int v = edge[i].to;
		if(v == far || v == HH) continue ;
		calc(v , u , val , dep + 1);
	}
}
void dsu(int u , int far , int op , int dep)
{
	for(int i = head[u] ; i ; i = edge[i].nex)
	{
		int v = edge[i].to;
		if(v == hson[u] || v == far) continue ;
		dsu(v , u , 0 , dep + 1);
	}
	if(hson[u]) dsu(hson[u] , u , 1 , dep + 1) , HH = hson[u];
	calc(u , far , 1 , dep);
	for(auto i : Q[u])
	{
		int b = i.fi , id = i.se , res = 0;
		rep(j , 0 , 25)
		{
			if(cnt[b][j] & 1) res ++ ;
		}
		if(res > 1) ans.pb(make_pair(id , 0));
		else ans.pb(make_pair(id , 1));
	}
	HH = 0;
	if(!op) calc(u , far , -1 , dep);
}
signed main()
{
	cin >> n >> m;
	rep(i , 2 , n)
	{
		int v;
		cin >> v;
		add_edge(i , v) , add_edge(v , i);
	}
	rep(i , 1 , n)
	{
		char ch;
		cin >> ch;
		col[i] = ch - 'a';
	}
	rep(i , 1 , m)
	{
		int a , b;
		cin >> a >> b;
		Q[a].pb(make_pair(b , i));
	}
	dfs(1 , 0);
	dsu(1 , 0 , 0 , 1);
	sort(ans.begin() , ans.end());
	for(auto i : ans) 
	if(i.se) cout << "Yes\n" ; 
	else cout << "No\n";
	return 0;
}
\end{lstlisting}

\end{document}
