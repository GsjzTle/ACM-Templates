\documentclass[E:/GsjzTle/main/main.tex]{subfiles}
\begin{document}

\textbf{题目大意}

\begin{quote}
一棵根为 \(1\) 的树,每条边上有一个字符(\(a-v\)共\(22\)种)

一条简单路径被称为Dokhtar-kosh当且仅当路径上的字符经过重新排序后可以变成一个回文串。

求每个子树中最长的Dokhtar-kosh路径的长度。
\end{quote}

\textbf{解题思路}

\begin{quote}
\(dsu\) \(on\) \(tree\) + 状态压缩

\begin{quote}
1.重排后构成回文的条件为:
\end{quote}

\begin{quote}
①.每个字母出现的次数都为偶数
②.一个字母出现次数为奇数,其余字母出现次数为偶数
\end{quote}

\begin{quote}
2.字母的范围为 a \textasciitilde{} z ,
把其转换成二进制状态(偶数为0,奇数为1)
\end{quote}

\begin{quote}
那么满足回文条件的二进制为 : 00...000 , 00..001 , 00..010 , ... ,
10..000
\end{quote}

\begin{quote}
3.维护一个从根节点到子节点u的前缀异或和数组 X
\end{quote}

\begin{quote}
那么 u 到 v 的简单路径的字母重排后的二进制形式为 X{[}u{]} \^{} X{[}v{]}
\end{quote}

\begin{quote}
4.节点 u 的答案有三种可能:
\end{quote}

\begin{quote}
①.它的两个不同分支的节点构成的简单路径
②.它的一个分支到它本身构成的简单路径 ③.它的子节点的 ans
\end{quote}

于是就可以定义 \(f_x\) 表示状态为 \(x\) 的节点的最大深度

那么当根节点为 \(rt\) 时 , 子节点 \(u\) 和 \(rt\) 产生的贡献为 ↓

\begin{lstlisting}
if((x[u] ^ x[rt]) == 0) ma = max(ma , dep[u] - dep[rt]); 
rep(i , 0 , 21) if((x[u] ^ x[rt]) == (1LL << i)) ma = max(ma , dep[u] - dep[rt]);
\end{lstlisting}

子节点 \(u\) 和 \(rt\) 的其它分支的产生贡献为 ↓

\begin{lstlisting}
if(f[x[u]]) ma = max(ma , f[x[u]] + dep[u] - 2 * dep[rt]);
rep(i , 0 , 21)
{
	int now = f[x[u] ^ (1LL << i)];
	if(now) ma = max(ma , dep[u] + now - 2 * dep[rt]);
}
\end{lstlisting}

\(rt\) 由它轻子儿子传递上来的贡献为 ↓

\begin{lstlisting}
for(int i = head[rt] ; i ; i = edge[i].nex)
{
	int v = edge[i].to;
	if(v == far || v == hson[rt]) continue ;
	dsu(v , rt , 0);
	ans[rt] = max(ans[rt] , ans[v]);
}
\end{lstlisting}

\(rt\) 由它重儿子传递上来的贡献为 ↓

\begin{lstlisting}
if(hson[rt]) dsu(hson[rt] , rt , 1) , ans[rt] = max(ans[rt] , ans[hson[rt]]) , HH = hson[rt];
if(f[x[rt]]) ma = max(ma , f[x[rt]] - dep[rt]);
rep(i , 0 , 21) if(f[x[rt] ^ (1LL << i)]) ma = max(ma , f[x[rt] ^ (1LL << i)] - dep[rt]);
\end{lstlisting}

因为要计算不同分支的两点产生的贡献,所以需要先对一个分支统计完贡献后,再添加它的信息

(事实上相同分支内的节点答案的在根节点为 \(rt\)
的子节点的时候就已经算过了)
\end{quote}

\begin{lstlisting}
const int N = 6e5 + 10;
struct Edge{
	int nex , to;
}edge[N << 1];
int head[N] , TOT;
void add_edge(int u , int v)
{
	edge[++ TOT].nex = head[u] ;
	edge[TOT].to = v;
	head[u] = TOT;
}
int hson[N] , HH , sz[N] , dep[N] , x[N];
int n , ma , a[N] , ans[N] , f[N * 20];
void dfs(int u , int far , int now)
{
	sz[u] = 1;
	dep[u] = dep[far] + 1;
	x[u] = now ^ a[u];
	for(int i = head[u] ; i ; i = edge[i].nex)
	{
		int v = edge[i].to;
		if(v == far) continue ;
		dfs(v , u , x[u]);
		sz[u] += sz[v];
		if(sz[v] > sz[hson[u]]) hson[u] = v;
	}
}
void calc(int u , int far , int rt)
{
	if(f[x[u]]) ma = max(ma , f[x[u]] + dep[u] - 2 * dep[rt]);
	if((x[u] ^ x[rt]) == 0) ma = max(ma , dep[u] - dep[rt]); 
	rep(i , 0 , 21)
	{
		if((x[u] ^ x[rt]) == (1LL << i)) ma = max(ma , dep[u] - dep[rt]);
		int now = f[x[u] ^ (1LL << i)];
		if(now) ma = max(ma , dep[u] + now - 2 * dep[rt]);
	}
	for(int i = head[u] ; i ; i = edge[i].nex)
	{
		int v = edge[i].to;
		if(v == far || v == HH) continue ;
		calc(v , u , rt);
	}
}
void change(int u , int far , int val)
{
	if(val == 1) f[x[u]] = max(dep[u] , f[x[u]]);
	else f[x[u]] = 0; 
	for(int i = head[u] ; i ; i = edge[i].nex)
	{
		int v = edge[i].to;
		if(v == far || v == HH) continue ;
		change(v , u , val);
	}
}
void dsu(int u , int far , int op)
{
	for(int i = head[u] ; i ; i = edge[i].nex)
	{
		int v = edge[i].to;
		if(v == far || v == hson[u]) continue ;
		dsu(v , u , 0);
		ans[u] = max(ans[u] , ans[v]);
	}
	if(hson[u]) dsu(hson[u] , u , 1) , HH = hson[u] , ans[u] = max(ans[u] , ans[hson[u]]);
	if(f[x[u]]) ma = max(ma , f[x[u]] - dep[u]);
	rep(i , 0 , 21) 
	{
		if(f[x[u] ^ (1LL << i)]) ma = max(ma , f[x[u] ^ (1LL << i)] - dep[u]);
	} 
	for(int i = head[u] ; i ; i = edge[i].nex)
	{
		int v = edge[i].to;
		if(v == far || v == hson[u]) continue ;
		calc(v , u , u);
		change(v , u , 1);
	}
	ans[u] = max(ans[u] , ma);
	f[x[u]] = max(f[x[u]] , dep[u]); 
	HH = 0;
	if(!op) 
	{
		ma = 0;
		change(u , far , -1);
	}
}
signed main()
{
	cin >> n;
	rep(i , 2 , n)
	{
		int x ; char op;
		cin >> x >> op;
		add_edge(i , x) , add_edge(x , i);
		a[i] = (1LL << (int)(op - 'a')); 
	}
	dfs(1 , 0 , 0);
	dsu(1 , 0 , 0);
	rep(i , 1 , n) cout << ans[i] << " \n"[i == n]; 
	return 0;
}
\end{lstlisting}

\end{document}
