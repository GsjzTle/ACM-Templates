\documentclass[E:/GsjzTle/main/main.tex]{subfiles}
\begin{document}

\textbf{过程}

\begin{itemize}
\item
  对于当前以 \(root\)
  为根的树,先去处理其轻子树,处理完轻子树后将轻子树所记录的信息删除,等轻子树全部处理完后再去处理其重子树,处理完重子树后将重子树所记录的信息保留,然后回到当前以
  \(root\) 为根的树,开始处理以 \(root\) 为根的树。
\item
  \(root\)
  保留了其重子树的信息,而没有轻子树的信息,因此需要再次遍历所有轻子树,并计算
  \textless{}\(root\),重子树\textgreater、\textless{}\(root\),轻子树\textgreater{}
  、\textless 重子树、轻子树\textgreater、\textless 轻子树,轻子树\textgreater(不同分会)之间所产生的贡献。等全部计算完后,判断
  \(root\) 是不是它爸爸的重儿子,如果是就保留 \(root\)
  及其全部子树的所有信息,若不是则删除 \(root\) 及其全部子树的所有信息。
\end{itemize}

\textbf{套路}

\begin{itemize}
\item
  如果计算贡献的式子中包含 \(dis(u,v)\),那么就把 \(dis(u,v)\) 拆解为
  \(dep_u+dep_v-2dep_{LCA}\)\\
  于是就可以以 \(LCA\) 为 \(root\),\(u,v\) 为其不同分支中的点,跑
  \(dsu~on~tree\) 计算贡献
\end{itemize}

\end{document}
