\documentclass[E:/GsjzTle/main/main.tex]{subfiles}

\begin{document}

\textbf{题目大意}

\begin{quote}
给定一片森林,每次询问一个节点的\(K-Son\)共有个多少不同的名字。一个节点的\(K-Son\)即为深度是该节点深度加\(K\)的节点。
\end{quote}

\textbf{解题思路}

\begin{quote}
比较裸的 dsu on tree

统计不同名字开个 \(map<string , int>\) 即可

也可以将字符串离散成数字再操作
\end{quote}

\begin{lstlisting}
const int N = 5e5 + 10;
struct Edge{
	int nex , to;
}edge[N << 1];
int head[N] , tot;
int a[N] , dep[N] , siz[N] , hson[N] , name[N] , HH;
vector<pair<int , int>>Q[N] , ans;
unordered_map<int , int>cnt[N] , sum;
void add_edge(int u , int v)
{
	edge[++ tot].nex = head[u];
	edge[tot].to = v;
	head[u] = tot;
}
void dfs(int u , int far)
{
	siz[u] = 1;
	dep[u] = dep[far] + 1;
	for(int i = head[u] ; i ; i = edge[i].nex)
	{
		int v = edge[i].to;
		if(v == far) continue ;		
		dfs(v , u);
		siz[u] += siz[v];
		if(siz[v] > siz[hson[u]]) hson[u] = v;
	}
} 
void calc(int u , int far , int val , int dep)
{
	cnt[dep][name[u]] += val;
	if(cnt[dep][name[u]] == 1 && val == 1) sum[dep] ++ ;
	if(!cnt[dep][name[u]]) sum[dep] -- ;
	for(int i = head[u] ; i ; i = edge[i].nex)
	{
		int v = edge[i].to;
		if(v == far || v == HH) continue ;
		calc(v , u , val , dep + 1);
	}
}
void dsu(int u , int far, int op , int dep)
{
	for(int i = head[u] ; i ; i = edge[i].nex)
	{
		int v = edge[i].to;
		if(v == far || v == hson[u]) continue ;
		dsu(v , u , 0 , dep + 1);
	}
	if(hson[u]) dsu(hson[u] , u , 1 , dep + 1) , HH = hson[u];
	calc(u , far , 1 , dep);
	for(auto i : Q[u])
	{
		int k = i.fi , id = i.se;
		ans.pb(make_pair(id , sum[dep + k]));
	}
	HH = 0;
	if(!op) calc(u , far , -1 , dep) , sum.clear();
}
map<string , int>vis;
signed main()
{
	int n , q , tot = 0;
	cin >> n;
	rep(i , 1 , n)
	{
		string s;
		int x;
		cin >> s >> x;
		if(!vis.count(s)) vis[s] = ++ tot;
		name[i] = vis[s] , a[i] = x;
		if(!x) continue ;
		add_edge(i , x) , add_edge(x , i);
	}
	rep(i , 1 , n) if(!a[i]) dfs(i , 0);
	cin >> q;
	rep(i , 1 , q)
	{
		int a , b;
		cin >> a >> b;
		Q[a].pb(make_pair(b , i));
	}
	rep(i , 1 , n)
	{
		if(!a[i]) dsu(i , 0 , 0 , 1) ;
	}
	sort(ans.begin() , ans.end());
	for(auto i : ans) cout << i.se << '\n';
	return 0;
}
\end{lstlisting}

\end{document}
