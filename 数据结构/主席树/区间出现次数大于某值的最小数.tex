\documentclass[E:/GsjzTle/main/main.tex]{subfiles}

\begin{document}

\textbf{查询区间内出现次数大于 (r - l + 1) / k 的最小数}

\begin{lstlisting}
const int N = 5e5 + 10;
struct Tree{
	int l , r , sum;
}tree[N * 40];
int root[N] , Cnt;
void update(int l , int r , int pre , int &now , int pos)
{
	tree[++ Cnt] = tree[pre];
	tree[Cnt].sum ++ ;
	now = Cnt;
	if(l == r) return ;
	int mid = l + r >> 1;
	if(pos <= mid) update(l , mid , tree[pre].l , tree[now].l , pos);
	else update(mid + 1 , r , tree[pre].r , tree[now].r , pos);
}
int query(int l , int r , int L , int R , int cnt)
{
	if(l == r) return l;
	int Lsum = tree[tree[R].l].sum - tree[tree[L].l].sum;
	int Rsum = tree[tree[R].r].sum - tree[tree[L].r].sum;
	int ans = 1e18 , mid = l + r >> 1;
	if(Lsum > cnt) ans = min(ans , query(l , mid , tree[L].l , tree[R].l , cnt));
	if(Rsum > cnt) ans = min(ans , query(mid + 1 , r , tree[L].r , tree[R].r , cnt));
	return ans;
}
int a[N];
vector<int>vec;
int get_id(int x)
{
	return lower_bound(all(vec) , x) - vec.begin() + 1;
}
signed main()
{
	ios::sync_with_stdio(false);
	cin.tie(0) , cout.tie(0);
	int n , m;
	cin >> n >> m;
	rep(i , 1 , n) cin >> a[i] , vec.pb(a[i]);
	sort(all(vec));
	vec.erase(unique(all(vec)) , vec.end());
	int nn = vec.size();
	rep(i , 1 , n) a[i] = get_id(a[i]) , update(1 , nn , root[i - 1] , root[i] , a[i]);
	while(m --)
	{
		int l , r , k;
		cin >> l >> r >> k;
		int res = query(1 , nn , root[l - 1] , root[r] , (r - l + 1) / k);
		if(res == 1e18) res = -1;
		else res = vec[res - 1];
		cout << res << '\n';
	}
	return 0;
} 
\end{lstlisting}

\end{document}
