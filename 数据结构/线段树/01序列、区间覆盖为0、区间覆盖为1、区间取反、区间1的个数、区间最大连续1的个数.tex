\documentclass[E:/GsjzTle/main/main.tex]{subfiles}

\begin{document}

\begin{itemize}
\item
  \texttt{0\ l\ r} 把 \([l, r]\) 区间内的所有数全变成 \(0\)
\item
  \texttt{1\ l\ r} 把 \([l, r]\) 区间内的所有数全变成 \(1\)
\item
  \texttt{2\ l\ r} 把 \([l,r]\) 区间内的所有数全部取反,也就是说把所有的
  \(0\) 变成 \(1\),把所有的 \(1\) 变成 \(0\)
\item
  \texttt{3\ l\ r} 询问 \([l, r]\) 区间内总共有多少个 \(1\)
\item
  \texttt{4\ l\ r} 询问 \([l, r]\) 区间内最多有多少个连续的 \(1\)
\end{itemize}

\begin{lstlisting}
#include <bits/stdc++.h>
#define ll long long
using namespace std;
const int MAX_N = 2e5 + 50;

int n, m, a[MAX_N];

struct segment_tree
{
	int l, r;
	int sum, lazy, rev, pre[2], suf[2], dat[2];
	#define ls(rt) (rt << 1)
	#define rs(rt) (rt << 1 | 1)
} t[MAX_N << 2];
void push_up(int rt)
{
	t[rt].sum = t[ls(rt)].sum + t[rs(rt)].sum;
	for (int i = 0; i <= 1; i ++ )
	{
		t[rt].pre[i] = t[ls(rt)].pre[i];
		if (i == 1 && t[ls(rt)].sum == (t[ls(rt)].r - t[ls(rt)].l + 1))
		t[rt].pre[i] += t[rs(rt)].pre[i];
		else if (i == 0 && t[ls(rt)].sum == 0)
		t[rt].pre[i] += t[rs(rt)].pre[i];
		
		t[rt].suf[i] = t[rs(rt)].suf[i];
		if (i == 1 && t[rs(rt)].sum == (t[rs(rt)].r - t[rs(rt)].l + 1))
		t[rt].suf[i] += t[ls(rt)].suf[i];
		else if (i == 0 && t[rs(rt)].sum == 0)
		t[rt].suf[i] += t[ls(rt)].suf[i];
		
		t[rt].dat[i] = t[ls(rt)].suf[i] + t[rs(rt)].pre[i];
		t[rt].dat[i] = max(t[rt].dat[i], t[ls(rt)].dat[i]);
		t[rt].dat[i] = max(t[rt].dat[i], t[rs(rt)].dat[i]);
		t[rt].dat[i] = max(t[rt].dat[i], t[rt].pre[i]);
		t[rt].dat[i] = max(t[rt].dat[i], t[rt].suf[i]);
	}
}
void push_down(int rt)
{
	if (t[rt].lazy != -1)
	{
		t[rt].rev = 0;
		int v = t[rt].lazy;
		
		t[ls(rt)].sum = (t[ls(rt)].r - t[ls(rt)].l + 1) * v;
		t[rs(rt)].sum = (t[rs(rt)].r - t[rs(rt)].l + 1) * v;
		
		t[ls(rt)].lazy = t[rs(rt)].lazy = v;
		t[ls(rt)].rev = t[rs(rt)].rev = 0;
		
		t[ls(rt)].pre[v] = t[ls(rt)].suf[v] = t[ls(rt)].dat[v] = (t[ls(rt)].r - t[ls(rt)].l + 1);
		t[ls(rt)].pre[v ^ 1] = t[ls(rt)].suf[v ^ 1] = t[ls(rt)].dat[v ^ 1] = 0;
		
		t[rs(rt)].pre[v] = t[rs(rt)].suf[v] = t[rs(rt)].dat[v] = (t[rs(rt)].r - t[rs(rt)].l + 1);
		t[rs(rt)].pre[v ^ 1] = t[rs(rt)].suf[v ^ 1] = t[rs(rt)].dat[v ^ 1] = 0;
		
		t[rt].lazy = -1;
	}
	if (t[rt].rev)
	{
		t[ls(rt)].sum = (t[ls(rt)].r - t[ls(rt)].l + 1) - t[ls(rt)].sum;
		t[rs(rt)].sum = (t[rs(rt)].r - t[rs(rt)].l + 1) - t[rs(rt)].sum;
		if (t[ls(rt)].lazy != -1) t[ls(rt)].lazy ^= 1;
		else t[ls(rt)].rev ^= 1;
		if (t[rs(rt)].lazy != -1) t[rs(rt)].lazy ^= 1;
		else t[rs(rt)].rev ^= 1;
		
		swap(t[ls(rt)].dat[0], t[ls(rt)].dat[1]);
		swap(t[ls(rt)].suf[0], t[ls(rt)].suf[1]);
		swap(t[ls(rt)].pre[0], t[ls(rt)].pre[1]);
		
		swap(t[rs(rt)].dat[0], t[rs(rt)].dat[1]);
		swap(t[rs(rt)].suf[0], t[rs(rt)].suf[1]);
		swap(t[rs(rt)].pre[0], t[rs(rt)].pre[1]);
		
		t[rt].rev = 0;
	}
	return;
}
void build (int rt, int l, int r)
{
	t[rt].l = l , t[rt].r = r , t[rt].lazy = -1;
	if (l == r)
	{
		t[rt].sum = a[l];
		t[rt].pre[a[l]] = t[rt].suf[a[l]] = t[rt].dat[a[l]] = 1;
		return;
	}
	int mid = (l + r) >> 1;
	build(ls(rt), l, mid);
	build(rs(rt), mid + 1, r);
	push_up(rt);
}
void update_range(int rt, int l, int r, int v)
{
	push_down(rt);
	if (l <= t[rt].l && t[rt].r <= r)
	{
		if (v == 0 || v == 1)
		{
			t[rt].sum = (t[rt].r - t[rt].l + 1) * v;
			t[rt].lazy = v;
			t[rt].pre[v] = t[rt].suf[v] = t[rt].dat[v] = (t[rt].r - t[rt].l + 1);
			t[rt].pre[v ^ 1] = t[rt].suf[v ^ 1] = t[rt].dat[v ^ 1] = 0;
		}
		else if (v == 2)
		{
			t[rt].sum = (t[rt].r - t[rt].l + 1) - t[rt].sum;
			t[rt].rev ^= 1;
			swap(t[rt].pre[0], t[rt].pre[1]);
			swap(t[rt].suf[0], t[rt].suf[1]);
			swap(t[rt].dat[0], t[rt].dat[1]);
		}
		return;
	}
	int mid = (t[rt].l + t[rt].r) >> 1;
	if (l <= mid) update_range(ls(rt), l, r, v);
	if (r > mid) update_range(rs(rt), l, r, v);
	push_up(rt);
}
ll query_sum(int rt, int l, int r)
{
	if (l <= t[rt].l && t[rt].r <= r) return t[rt].sum;
	push_down(rt);
	int mid = (t[rt].l + t[rt].r) >> 1;
	int ans = 0;
	if (l <= mid) ans += query_sum(ls(rt), l, r);
	if (r > mid) ans += query_sum(rs(rt), l, r);
	return ans;
}
segment_tree query_len(int rt, int l, int r)
{
	if (l <= t[rt].l && t[rt].r <= r) return t[rt];
	push_down(rt);
	int mid = (t[rt].l + t[rt].r) >> 1;
	if (r <= mid) return query_len(ls(rt), l, r);
	if (l > mid) return query_len(rs(rt), l, r);
	segment_tree t1 = query_len(ls(rt), l, r), t2 = query_len(rs(rt), l, r), t3;
	t3.sum = t1.sum + t2.sum;
	for (int i = 0; i <= 1; i ++ )
	{
		t3.pre[i] = t1.pre[i];
		if (i == 1 && t1.sum == (t1.r - t1.l + 1))
		t3.pre[i] += t2.pre[i];
		else if (i == 0 && t1.sum == 0)
		t3.pre[i] += t2.pre[i];
		
		t3.suf[i] = t2.suf[i];
		if (i == 1 && t2.sum == (t2.r - t2.l + 1))
		t3.suf[i] += t1.suf[i];
		else if (i == 0 && t2.sum == 0)
		t3.suf[i] += t1.suf[i];
		
		t3.dat[i] = t1.suf[i] + t2.pre[i];
		t3.dat[i] = max(t3.dat[i], t1.dat[i]);
		t3.dat[i] = max(t3.dat[i], t2.dat[i]);
		t3.dat[i] = max(t3.dat[i], t3.pre[i]);
		t3.dat[i] = max(t3.dat[i], t3.suf[i]);
	}
	return t3;
}
signed main()
{
	cin >> n >> m;
	for (int i = 1; i <= n; i ++ ) cin >> a[i];
	build(1, 1, n);
	for (int i = 1; i <= m; i ++ )
	{
		int op , l , r;
		cin >> op >> l >> r;
		l ++ , r ++ ;  // 本题下标是从 0 开始的 
		if (op == 0) update_range(1, l, r, 0);
		if (op == 1) update_range(1, l, r, 1);
		if (op == 2) update_range(1, l, r, 2);
		if (op == 3) cout << query_sum(1, l, r) << '\n';
		if (op == 4) cout << query_len(1, l, r).dat[1] << '\n';
	}
	return 0;
}
\end{lstlisting}

\end{document}
