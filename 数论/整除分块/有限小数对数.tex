\documentclass[E:/GsjzTle/main/main.tex]{subfiles}

\begin{document}

\begin{itemize}
\item
  求出有序整数对 \((x,y)\) 的个数,使得 \(1\le x,y\le n\) 且
  \(\frac{x}{y}\) 可以表示为十进制有限小数。
\item
  \(1 \le n \le 10^{12}\)
\end{itemize}

对于一个有限小数,当分数为最简形式时分母只包含 2 、5
两个质因子,所以我们可以用 \(\lfloor\frac{bc}{ac}\rfloor\)
来表示任何一个有限小数(其中 \(a\) 为 \(2,5\) 两个质因子构成的数 , \(c\)
为不包含 \(2,5\) 两个质因子的数 , \(b\) 随意)。\\
于是这道题就可以枚举 \(c\),那么
\(\text { ans }=\sum_{c=1}^{n} \lfloor\frac{n}{c}\rfloor \times f\left(\lfloor\frac{n}{c}\rfloor\right)\)(其中
\(\lfloor\dfrac{n}{c}\rfloor\) 为 \(b\)
的取值范围,\(f(\lfloor\frac{n}{c}\rfloor)\))为
\([1,\lfloor\frac{n}{c}\rfloor]\) 内 \(a\) 的个数。\\
不过值得注意的是这里的 \(c\) 是不包含 \(2,5\) 两个质因子的数 ,
所以对于每个块我们要减去包含 \(2,5\) 两个质因子的数。实现起来也很简单 ,
根据容斥原理减去块内 \(2\) 的倍数的数 , 再减去块内 \(5\) 的倍数的数 ,
然后再加上块内 \(10\) 的倍数的数即可。\\
至于 \(f(\lfloor\frac{n}{c}\rfloor)\) ,我们可以先用 \(log\)
的复杂度求出所有 \(1\sim10^{12}\) 的所有 \(a\),并记录 \(a\) 的个数\\
因为随着 \(c\) 增大 \(f(\lfloor\frac{n}{c}\rfloor)\)
是递减的,所以可以用单指针求出

\begin{lstlisting}
const int N = 3e5 + 10;
int a[N] , m , now = 1; 
int f(int x)
{
	while(a[now] > x && now ) now -- ;
	return now ;
}
signed main()
{
	int n ;
	cin >> n;
	for(int i = 1 ; i <= n ; i *= 2)
	for(int j = 1 ; j * i <= n ; j *= 5)
	a[++ m] = i * j;
	now = m;
	sort(a + 1 , a + 1 + m);
	int ans = 0;
	for(int l = 1 , r ; l <= n ; l = r + 1)
	{
		r = n / (n / l);
		int cnt1 = r - l + 1 , cnt10 = r / 10 - (l - 1) / 10;
		int cnt2 = r / 2 - (l - 1) / 2 , cnt5 = r / 5 - (l - 1) / 5;
		int sum = cnt1 - cnt2 - cnt5 + cnt10;
		ans += sum * (n / l) * f(n / l);
	}
	cout << ans << '\n';
	return 0;
}
\end{lstlisting}

\end{document}
