\documentclass[E:/GsjzTle/main/main.tex]{subfiles}

\begin{document}

\begin{itemize}
\item
  \textbf{模型:}\(\sum_{i=1}^{n}\left\lfloor\frac{n}{i}\right\rfloor\)\\
  假设 \(n = 8\),那么可得:
\end{itemize}

\begin{longtable}[]{@{}lllllllll@{}}
\toprule
\(i\) & 1 & 2 & 3 & 4 & 5 & 6 & 7 & 8 \\
\midrule
\endhead
\(8/i\) & 8 & 4 & 2 & 2 & 1 & 1 & 1 & 1 \\
\bottomrule
\end{longtable}

\begin{itemize}
\item
  \textbf{概念:}\\
  表中同样的值会连续出现,而相同的值所划分的区间成为一个块。\\
  整除的性质使得从 \(1\) 到 \(n\)
  的表可根据数值划分为不同的块,且分块数远远小于 \(n\)。\\
  利用这种性质,我们可以推导出每个分块具体的左右端点位置在哪,这样就可以快速求解出来了。\\
  复杂度 \(O(\sqrt{n})\)
\item
  \textbf{推导:}\\
  假设我们已知某一个分块的左端点 \(l\),要求解出该分块的右端点 \(r\)。\\
  设该分块的数值为\(k\),对于该分块中的每个数\(i\),有
  \(k=\left\lfloor\frac{n}{i}\right\rfloor=\left\lfloor\frac{n}{l}\right]\),即
  \(i\times k \le n\)。\\
  那么显然满足 \(i \times k \le n\) 成立的最大的 \(i\)
  就是我们要的右端点 \(r\)。\\
  于是可得:\(\left\{\begin{array}{l}k=\left\lfloor\frac{n}{l}\right\rfloor \\ r=\max (i), i \times k \leq n\end{array}\right.\)\\
  推导得:\(r=\left\lfloor\frac{n}{k}\right\rfloor=\left\lfloor \frac{n}{\left\lfloor\frac{n}{l}\right\rfloor}\right\rfloor\)
\item
  \textbf{变形0:}\\
  已知 \(n,m\),求
  \(\sum_{i=1}^{min(n,m)}\lfloor\frac{n}{ i}\rfloor\lfloor\frac{m}{i}\rfloor\)\\
  \(\left\{\begin{array}{l}k1=\left\lfloor\frac{n}{l}\right\rfloor,k2= \left\lfloor\frac{m}{l}\right\rfloor\\ r=\max (i), i \times k1 \leq n,i\times k2 \le m\end{array}\right.\)\\
  \(r = min(\lfloor\frac{n}{k1}\rfloor,\lfloor\frac{m}{k2}\rfloor)\)
\item
  \textbf{变形1:}

  已知 \(n,a,b\),求
  \(\sum_{i=1}^{n}\left\lfloor\frac{n}{a i+b}\right\rfloor\)\\
  令 \(i' = a\times i + b\),先求出
  \(\left\lfloor\frac{n}{i'}\right\rfloor\) 的整除分块\\
  可得
  \(k=\left\lfloor\frac{n}{a\times l+b}\right\rfloor\),\(r'=\left\lfloor\frac{n}{k}\right\rfloor=\left\lfloor\frac{n}{\left\lfloor\frac{n}{a \times l + b}\right\rfloor}\right\rfloor\)\\
  \(\left\{\begin{array}{l}i'=a\times i +b \\ r'=\max (i')\end{array}\right.\)→\(\begin{cases}a\times i=i'-b\\
  i=\dfrac{i'-b}{a}\end{cases}\)→
  \(r=max(i)= max(\left\lfloor\frac{i'-b}{a}\right\rfloor) = \lfloor\frac{r'-b}{a} \rfloor\)\\
  \(r =\left \lfloor \dfrac{\left \lfloor \dfrac{n}{\lfloor \dfrac{n}{a\times l+b}\rfloor }\right \rfloor -b}{a}\right \rfloor \)
\item
  \textbf{变形2:}

  做法同上。\\
  已知 \(n\),求
  \(\sum_{i=1}^{n}\left\lfloor\frac{n}{i^2}\right\rfloor\)\\
  令 \(i' = i^2\),先求出 \(\left\lfloor\frac{n}{i'}\right\rfloor\)
  的整除分块\\
  可得
  \(k=\left\lfloor\frac{n}{l^2}\right\rfloor\),\(r'=\left\lfloor\frac{n}{k}\right\rfloor=\left\lfloor\frac{n}{\left\lfloor\frac{n}{l^2}\right\rfloor}\right\rfloor\)\\
  \(\left\{\begin{array}{l}i=\sqrt{i'} \\ r'=\max (i')\end{array}\right.\)
  →
  \(r = max(i) = max(\lfloor\sqrt{i'} \rfloor) = \lfloor r' \rfloor = \lfloor\sqrt{\left\lfloor\frac{n}{\left\lfloor\frac{n}{l^2}\right\rfloor}\right\rfloor}\rfloor\)
\item
  \textbf{变形3:}\\
  已知 \(n\),求 \(\sum_{i=1}^{n}\left\lceil\frac{n}{i}\right\rceil\)\\
  当 \(n\) 整除 \(i\)
  时,\(\left\lceil\frac{n}{i}\right\rceil = \left\lfloor\frac{n}{i}\right\rfloor\)\\
  当 \(n\) 不整除 \(i\)
  时,\(\left\lceil\frac{n}{i}\right\rceil = \left\lfloor\frac{n}{i}\right\rfloor + 1\)\\
  于是我们可以用 \(\lfloor \frac{n+i-1}{i}\rfloor \) 来代替
  \(\lceil\frac{n}{i}\rceil\)\\
  原式就转换为了 \(\sum_{i=1}^{n}\lfloor\frac{n+i -1}{i}\rfloor\)

  那么有
  \(\left\{\begin{array}{l}k=\left\lfloor\frac{n+l-1}{l}\right\rfloor \\ r=\max (i), i\times k \leq n+i-1 \end{array}\right.\)\\
  \(\because i \times k \le n+i-1 \Rightarrow  i\times(k-1) \le n-1\)\\
  \(\therefore r=\left\lfloor\frac{n-1}{k-1}\right\rfloor=\left[\frac{n-1}{\left\lfloor\frac{n+l-1}{l} \mid-1\right.}\right\rfloor\)
\end{itemize}

\end{document}
