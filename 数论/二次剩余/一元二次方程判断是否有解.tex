\documentclass[E:/GsjzTle/main/main.tex]{subfiles}

\begin{document}

对于一个模意义下的一元二次方程:\(x^2 + bx + c = 0 (\bmod p)\),其中
\(p\) 是质数。\\
每次给定一组 \(b,c,p\) (\(a = 1\)),问这个方程有没有整数解,有解输出
\(Yes\),无解输出 \(No\)

\begin{itemize}
\item
  对于一般的二次同余方程\\
  \(a x^{2}+b x+c=0(\bmod p)\)
\item
  可以通过配方化为下式:\\
  \((2 a x+b)^{2}=b^{2}-4 a c(\bmod 4 a p)\)
\item
  设 \(X=2 a x+b(\bmod 4 a p)\)
\item
  解方程 \(X^{2}=b^{2}-4 a c(\bmod 4 a p)\)
\item
  可以先解出 \(X^{2}=b^{2}-4 a c(\bmod p)\) 的解,再去不断 \(+p\) 凑
  \(4ap\),但经过进一步讨论,发现 \(p>2\) 时
  \(X^{2}=b^{2}-4 a c(\bmod p)\)
  有解、\(X^{2}=b^{2}-4 a c(\bmod 4 a p)\) 一定有解(证明略)
\item
  当 \(p\) 为 \(2\) 时,直接特判
\end{itemize}

\begin{lstlisting}
#include <bits/stdc++.h>
#define ll long long
using namespace std;
ll pow_mod(ll a, ll i, ll n)
{
	if(!i) return 1 % n;
	ll temp = pow_mod(a, i >> 1, n);
	temp = temp * temp % n;
	if(i & 1) temp = temp * a % n;
	return temp;
}
ll modsqr(int a, int n)
{
	ll b, k, i, x;
	if(n == 2) return a % n;
	if(pow_mod(a, (n - 1) / 2, n) == 1)
	{
		if(n % 4 == 3) x = pow_mod(a, (n + 1) / 4, n);
		else
		{
			for(b = 1; b = pow_mod(b, (n - 1) / 2, n), n == 1; b ++) ;
			i = (n - 1) / 2 , k = 0;
			do
			{
				i /= 2 , k /= 2;
				if ((pow_mod(a, i, n) * (ll)pow_mod(b, k, n) + 1) % n == 0)
				{
					k += (n - 1) / 2;
				}
			} while (i % 2 == 0);
			x = (pow_mod(a, (i - 1) / 2, n) * (ll)pow_mod(b, k / 2, n)) % n;
		}
		if(x * 2 > n) x = n - x;
		return x;
	}
	if(a == 0) return 0;
	return -1;
}
signed main()
{
	int T = 1;
	cin >> T;
	while (T--)
	{
		ll a = 1 , b, c , p;
		cin >> b >> c >> p;
		if (p == 2)
		{
			if (c % p == 0 || (a + b + c) % p == 0) puts("Yes");
			else puts("No");
			continue;
		}
		ll t = b * b - 4 * a * c;
		t = (t % p + p) % p;
		ll s = modsqr((ll)t, p);
		bool flag = (s != -1);
		puts(flag ? "Yes" : "No");
	}
	return 0;
}
\end{lstlisting}

\end{document}
