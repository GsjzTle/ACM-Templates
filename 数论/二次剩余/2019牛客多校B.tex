\documentclass[E:/GsjzTle/main/main.tex]{subfiles}

\begin{document}

已知 \(0<=x<=y<p,p=1e9+7\),且有

\begin{itemize}
\item
  \((x~+~y)=b~\bmod~p\)
\item
  \((x~\times~y)=c~\bmod~p\)
\item
  求解任意一对\(x,y\),不存在输出\(−1 −1\)
\end{itemize}

由两式变化可得
\((y-x)^{2}=\left(b^{2}-4 c+p\right) \% p\bmod p\),那么可以应用二次剩余定理解得
\(y−x\) 的值\\
由题我们可以知道\((x+y)=b\) 或者 \((x+y)=b+p\),那么直接求解即可。

\begin{lstlisting}
#include<bits/stdc++.h>
#define ll long long
using namespace std;
const ll MOD = 1e9 + 7;
ll ppow(ll a, ll b, ll mod)
{
	ll ret = 1;
	a = a % mod;
	while(b)
	{
		if(b & 1) ret = ret * a % mod;
		a = a * a % mod;
		b >>= 1;
	}
	return ret;
}
struct TT
{
	ll p, d;
};
ll w;
TT mul_er(TT a, TT b, ll mod)
{
	TT ans;
	ans.p = (a.p * b.p % mod + a.d * b.d % mod * w % mod) % mod;
	ans.d = (a.p * b.d % mod + a.d * b.p % mod) % mod;
	return ans;
}
TT power(TT a, ll b, ll mod)
{
	TT ret;
	ret.p = 1, ret.d = 0;
	while(b)
	{
		if(b & 1) ret = mul_er(ret, a, mod);
		a = mul_er(a, a, mod);
		b >>= 1;
	}
	return ret;
}
ll legendre(ll a, ll p)
{
	return ppow(a, (p - 1) >> 1, p);
}
ll modulo(ll a, ll mod)
{
	a %= mod;
	if(a < 0) a += mod;
	return a;
}
ll solve(ll n, ll p)    //x^2 = n mod p
{
	if(n == 0) return 0;
	if(n == 1) return 1;
	if(p == 2) return 1;
	if(legendre(n, p) + 1 == p) return -1;  //无解
	ll a = -1, t;
	while(true)
	{
		a = rand() % p;
		t = a * a - n;
		w = modulo(t, p);
		if(legendre(w, p) + 1 == p) break;
	}
	TT temp;
	temp.p = a;
	temp.d = 1;
	TT ans = power(temp, (p + 1) >> 1, p);
	return ans.p;
}
bool getans(ll sum, ll dec, ll &x, ll &y)
{
	if((sum + dec) % 2 == 0)
	{
		y = (sum + dec) / 2;
		x = y - dec;
		if(x >= 0 && x + y == sum && y < MOD) return true;
		else return false;
	}
	else return false;
}
signed main()
{
	int T;
	cin >> T;
	while(T--)
	{
		ll b, c;
		cin >> b >> c;
		ll d = solve((b * b % MOD - 4 * c % MOD + MOD) % MOD, MOD);
		if(d == -1)
		{
			cout << "-1 -1\n";
			continue;
		}
		ll x, y;
		if(getans(b, d, x, y)) cout << x << " " << y << '\n';
		else if(getans(b + MOD, d, x, y)) cout << x << " " << y << '\n';
		else if(getans(b, MOD - d, x, y)) cout << x << " " << y << '\n';
		else if(getans(b + MOD, MOD - d, x, y)) cout << x << " " << y << '\n';
	}
	return 0;
}
\end{lstlisting}

\end{document}
