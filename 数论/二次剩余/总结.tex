\documentclass[E:/GsjzTle/main/main.tex]{subfiles}
\begin{document}

\textbf{什么是二次剩余?}

\begin{quote}
给出一个式子 \(x^2\equiv n\),再给出 \(n\) 和 \(p\),如果能求得一个
\(x\) 满足该式子,即 \(x^2 = n + kp,k∈z\),那么我们称 \(n\) 是模 \(p\)
的二次剩余。若不存在这样的 \(x\),则我们称 \(n\) 是模 \(p\)
的非二次剩余。同时我们称 \(x\) 为该二次同余方程的解。
\end{quote}

\textbf{二次剩余的作用?}

\begin{quote}
对于一个数 \(n\),如果我们要求 \(\sqrt{n}\bmod p\)的值,那么我们可以看
\(n\) 是否是模 \(p\) 的二次剩余,如果是的话就会满足
\(x^2\equiv n(\bmod p)\rightarrow x\equiv \sqrt{n}(\bmod p)\),那么我们就可以用
\(x\) 来代替 \(\sqrt{n}\),即只需要求该二次同余方程的解即可。\\
说白了就是如果该二次同余方程有解,那么 \(n\) 可以在模 \(p\)
的意义下开根号
\end{quote}

\textbf{二次同余方程定理}

\begin{quote}
\begin{itemize}
\item
  定理1:对于\(x^2\equiv n(\bmod p)\),总共有\(\dfrac{p-1}{2}\)个的
  \(n\) 能使该方程有解(将 \(n=0\) 情况除去,由于该情况显然有\(x=0\))
\item
  定理2:\(\left(\frac{a}{p}\right)=\left\{\begin{array}{ll}
  1, & a \text { 在模 } p \text { 意义下是二次剩余 } \\
  -1, & a \text { 在模 } p \text { 意义下是非二次剩余 } \\
  0, & a \equiv 0(\bmod p)
  \end{array}\right.\)

  \((\dfrac{a}{p})\equiv a^{\dfrac{p-1}{2}}(\bmod p)\)
\item
  定理3:略
\item
  定理四:\((a+b)^{n} \equiv a^{n}+b^{n}(\bmod n)(n \in P)\)
\end{itemize}
\end{quote}

\end{document}
