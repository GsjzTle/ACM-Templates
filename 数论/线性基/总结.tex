\documentclass[E:/GsjzTle/main/main.tex]{subfiles}

\begin{document}

\textbf{线性基是啥?}

\begin{itemize}
\item
  线性基是一个\textbf{数的集合},并且每个序列都拥有\textbf{至少一个}线性基
\end{itemize}

\textbf{线性基三大性质}

\begin{enumerate}
\def\labelenumi{\arabic{enumi}.}
\item
  原序列里面的任意一个数都可以由线性基里面的一些数异或得到
\item
  线性基里面的任意一些数异或起来都不能得到 \(0\)
\item
  线性基里面的数的个数唯一,并且在保持性质一的前提下,数的个数是最少的
\end{enumerate}

\textbf{关于性质三的证明:}

\textbf{假如序列里面的所有元素都可以插入到线性基里面}\\
显然如果是这种情况的话,不管是用什么顺序将序列里的数插入到线性基里,线性基中的元素一定与原序列元素数量相同。所以性质\(3\)成立。

\textbf{假如序列里面的一些元素不能插入到线性基里面}\\
我们设x不能插入到线性基里面,那么一定满足形如
\(d[a] \oplus d[b] \oplus d[c]=x\) 的式子\\
那我们尝试将插入顺序改变,变成:\(d[a]、d[b]、x、d[c]\)
,那么显然,\(d[c]\) 是不可能插入成功的,简单的证明:

\(∵ d[a] \oplus d[b] \oplus d[c] = x\)\\
\(∴ d[a] \oplus d[b] \oplus x = d[c]\)(根据上面那条并没有什么卵用的异或性质)\\
原来是 \(x\) 插入不进去,改变顺序后,\(d[c]\)
插入不进去,也就是说,对于插入不进去的元素,改变插入顺序后,要么还是插入不进去,要么就是插入进去了\\
同时另\textbf{一个}原来插入的进去的元素插入不进去了,所以,\textbf{可以插入进去的元素数量一定是固定的}\\
\textbf{不可能因为一个元素而导致多个元素不能插入,就算去掉这个元素,也只能插入那多个元素中的一个!}\\
如果你去掉线性基里面的任意一个数,都会使得原序列里的一些(或一个)数无法通过用线性基里的元素异或得到\\
所以,每一个元素都是必要的,换句话说,这里面没有多余的元素,所以,这个线性基的元素个数在保持性质\(1\)的前提下,一定是最少的。

\end{document}
