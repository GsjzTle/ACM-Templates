\documentclass[E:/GsjzTle/main/main.tex]{subfiles}
\begin{document}

问题:

\begin{quote}
给定一个长度为 \(N\) 的数组 \(A\),\(M\) 次询问和一个常数 \(K\)\\
每次询问给出区间 \([L,R]\),要求在区间中找出若干个数\\
使得这些数的异或和或上 \(K\) 的值最大,即
\texttt{ans\ =\ MAX(sum\ \textbar{}\ K)}
\end{quote}

实现:线段树套线性基

思路:

\begin{quote}
区间操作 ------ 线段树\\
异或操作 ------ 线性基

首先 \(ans=K|sum\),所以并不是 \(sum\)(异或和) 越大,\(ans\) 就越大

举个例子:\(K = 1000\),\(sum1 = 1000\),\(sum2 = 0111\)\\
那么显然 \(sum1 > sum2\),\((K|sum1) < (K|sum2)\)

我们发现按位或运算,只要有一个 \(1\) 就行了,两个 \(1\) 和一个 \(1\)
没有区别,所以对于 \(K\) 中二进制下为
\(1\)的所有位,线性基得到的\(sum\),这一位都没必要是1,所以我们可以把整个\(A\)
数组的 \(K\) 是 \(1\) 的所有二进制位全部变成 \(0\)。

这样等于删掉了不需要的 \(1\) ,因为这些 \(1\)
存在在这个线性基里时,会导致最大异或和更大,但是这个更大的数对我们最终的答案没有贡献。

所以我们可以将 \(k\)
取反,然后把所有数在加入线性基之前,全部按位异或运算一遍,这样把没用的
\(1\) 全部删掉,再加入线性基

这样求出来的 \(max_sum\)(最大异或和),每一位都能使得 \(K\)
变为更大的数了
\end{quote}

\begin{lstlisting}
const int N = 50007, mod = 1e9 + 7, INF = 2.1e9;
const int Base = 27 * 2;//线性基数组大小,开大一点,嘻嘻嘻
//s < 2 ^ {63} - 1
//说明线性基最多有62个向量
struct LinearBase
{
	int dimension = 27; // dimension 维数,就是线性基的维数 = logs, (s为元素最大值)
	//dimension一定要是logs,错一点就会WA呜呜呜
	//线性基数组
	ll a[Base + 7];//注意这里要加一点,因为下面的循环也是这个数
	LinearBase()
	{
		fill(a, a + Base + 7, 0);
	}
	LinearBase(ll *x, int n)
	{
		build(x, n);
	}
	void insert(ll t)
	{
		// 逆序枚举二进制位
		for(int i = dimension; i >= 0; -- i)
		{
			if(t == 0) return ;
			// 如果 t 的第 i 位为 0,则跳过
			if(!(t >> i & 1)) continue;
			// 如果 a[i] != 0,则用 a[i] 消去 t 的第 i 位上的 1
			if(a[i]) t ^= a[i];
			else
			{
				// 找到可以插入 a[i] 的位置
				// 用 a[0...i - 1] 消去 t 的第 [0, i) 位上的 1
				// 如果某一个 a[k] = 0 也无须担心,因为这时候第 k 位
				//不存在于线性基中,不需要保证 t 的第 k 位为 0
				for(int k = 0; k < i; ++ k)
				if(t >> k & 1) t ^= a[k];
				// 用 t 消去 a[i + 1...L] 的第 i 位上的 1
				for(int k = i + 1; k <= dimension; ++ k)
				if(a[k] >> i & 1) a[k] ^= t;
				// 插入到 a[j] 的位置上
				a[i] = t;
				break;
			}
			// 此时 t 的第 i 位为 0,继续寻找其最高位上的 1
		}
		// 如果没有插入到任何一个位置上,则表明 t 可以由 a 中若干个元素的异或和表示出,即 t 在 span(a) 中
	}
	// 数组 x 表示集合 S,下标范围 [1...n]
	void build(ll *x, int n)
	{
		fill(a, a + Base + 7, 0);
		for(int i = 1; i <= n; ++ i)
		insert(x[i]);
	}
	ll query_max()
	{
		ll res = 0;
		for(int i = 0; i <= dimension; ++ i)
		res ^= a[i];
		return res;
	}
	void mergefrom(const LinearBase &other)
	{
		for(int i = 0; i <= dimension; ++ i)
		insert(other.a[i]);
	}
	static LinearBase merge(const LinearBase &a, const LinearBase &b)
	{
		LinearBase res = a;
		for(int i = 0; i <= 27; ++ i)
		res.insert(b.a[i]);
		return res;
	}
};
int n, m, q, k;
ll a[N];
struct Tree
{
	int l, r;
	LinearBase elem;//element
} tr[N];
void build(int p, int l, int r)
{
	tr[p] = {l, r};
	if(l == r)
	{
		tr[p].elem.insert(a[r]);
		return ;
	}
	int mid = (l + r) >> 1;
	build(p << 1, l, mid);
	build(p << 1 | 1, mid + 1, r);
	tr[p].elem = tr[p].elem.merge(tr[p << 1].elem, tr[p << 1 | 1].elem);
}
LinearBase query(int p, int l, int r)
{
	if(l <= tr[p].l && tr[p].r <= r) return tr[p].elem;
	int mid = (tr[p].l + tr[p].r) >> 1;
	LinearBase res;
	if(l <= mid) res = res.merge(res, query(p << 1, l, r));
	if(r > mid) res = res.merge(res, query(p << 1 | 1, l, r));
	return res;
}
signed main()
{
	ios::sync_with_stdio(false);
	cin.tie(0) , cout.tie(0);
	int T = 1;
	cin >> T;
	while(T --)
	{
		cin >> n >> q >> k ;
		rep(i , 1 , n) 
		{
			cin >> a[i];
			/*k的存在会对求线性基最大值时的主元产生影响,所以预处理一下,a[i]只保留k为0的位,这样贡献最大*/
			for(int j = 0; j < 27; ++ j)
			{
				if((k >> j & 1) && (a[i] >> j & 1))
				a[i] ^= (1 << j);
			}
		}
		build(1, 1, n);
		while(q -- )
		{
			int l, r;
			cin >> l >> r;
			LinearBase res = query(1, l, r);
			cout << (res.query_max() | k) << '\n';
		}
	}
	return 0;
}
\end{lstlisting}

\end{document}
