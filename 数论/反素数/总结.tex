\documentclass[E:/GsjzTle/main/main.tex]{subfiles}

\begin{document}

\textbf{定义}

\begin{itemize}
\item
  对于任何正整数 \(x\),其约数的个数记作 \(g(x)\),例如
  \(g(1)=1\),\(g(6)=4\)
\item
  如果某个正整数 \(x\) 满足:\(\forall 0 < i < x\),都有
  \(g(x) > g(i)\),则称 \(x\) 为 \textbf{反素数},例如整数 \(1,2,4,6\)
  等都是反质数
\end{itemize}

\textbf{性质1}

\begin{itemize}
\item
  \(1\) \textasciitilde{} \(N\) 中的最大的反质数,就是
  \(1\)\textasciitilde{}\(N\) 中约数个数最多的数中最小的一个
\end{itemize}

\textbf{证明1}

\begin{quote}
设 \(m\) 是\(1\) \textasciitilde{} \(N\)
中约数个数最多的数中最小的一个\\
根据 \(m\) 的定义,\(m\) 显然满足:\\
1.\({\forall}x < m, g(x) < g(m)\)\\
2.\({\forall} x > m , g(x) ≤ g(m)\)\\
根据反素数的定义,\(1\) 说明 \(m\) 是反质数,\(2\) 说明大于 \(m\)
的数都不是反质数,证毕
\end{quote}

\textbf{性质2}

\begin{itemize}
\item
  \(1\) \textasciitilde{} \(N\) 中任何数的不同质因子都不会超过 \(10\)
  个,且所有质因子的指数总和不超过 \(30\) (前提是 \(N <= 2\times10^9\))
\item
  证明略
\end{itemize}

\textbf{性质3}

\begin{itemize}
\item
  \(x\) 为反质数的必要条件是:\(x\) 分解质因数后可写作
  \(2^{c1}\times3^{c2}\times...\) ,并且 \(c1≥c2≥c3≥...≥0\)
\item
  证明略
\end{itemize}

\textbf{性质4}

\begin{itemize}
	\item
	一个反素数位全部倒过来后还是一个反素数,比如 \(13\)、\(31\) 都是反素数,\(17\)、\(71\) 都是反素数... 
\end{itemize}

\end{document}
