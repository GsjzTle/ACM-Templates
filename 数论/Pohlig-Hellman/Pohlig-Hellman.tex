\documentclass[E:/GsjzTle/main/main.tex]{subfiles}

\begin{document}

\begin{lstlisting}
namespace PhoRho   //Phollard-Rho
{
	ll gcd(ll a, ll b)
	{
		if (b == 0) return a;
		return gcd(b, a % b);
	}
	ll fastpow(ll x, ll p, ll mod)
	{
		ll ans = 1;
		while (p)
		{
			if (p & 1) ans = (__int128) ans * x % mod;
			x = (__int128) x * x % mod;
			p >>= 1;
		}
		return ans;
	}
	ll max_factor;
	bool MillerRabin(ll x, ll b)
	{
		ll k = x - 1;
		while (k)
		{
			ll cur = fastpow(b, k, x);
			if (cur != 1 && cur != x - 1) return false;
			if ((k & 1) == 1 || cur == x - 1) return true;
			k >>= 1;
		}
		return true;
	}
	bool prime(ll x)
	{
		if (x == 46856248255981ll || x < 2) return false;
		if (x == 2 || x == 3 || x == 7 || x == 61 || x == 24251) return true;
		return MillerRabin(x, 2) && MillerRabin(x, 3) && MillerRabin(x, 7) && MillerRabin(x, 61);
	}
	ll f(ll x, ll c, ll n)
	{
		return ((__int128) x * x + c) % n;
	}
	ll PRho(ll x)
	{
		ll s = 0, t = 0, c = 1ll * rand() * (x - 1) + 1;
		int stp = 0, goal = 1;
		ll val = 1;
		for (goal = 1;; goal <<= 1, s = t, val = 1)
		{
			for (stp = 1; stp <= goal; ++stp)
			{
				t = f(t, c, x);
				val = (__int128) val * abs(t - s) % x;
				if (stp % 127 == 0)
				{
					ll d = gcd(val, x);
					if (d > 1) return d;
				}
			}
			ll d = gcd(val, x);
			if (d > 1) return d;
		}
	}
	void fac(ll x)
	{
		if (x <= max_factor || x < 2)
		{
			return;
		}
		if (prime(x))
		{
			max_factor = max_factor > x ? max_factor : x;
			return;
		}
		ll p = x;
		while (p >= x) p = PRho(x);
		while ((x % p == 0)) x /= p;
		fac(x);
		fac(p);
	}
	ll divide(ll n)
	{
		srand((unsigned) time(0));
		max_factor = 0;
		fac(n);
		return max_factor;
	}
}
namespace DLP
{
	const int N = 1111111;
	ll fastpow(ll a, ll n)   //快速幂
	{
		ll res = 1;
		while (n > 0)
		{
			if (n & 1) res = res * a;
			a = a * a;
			n >>= 1;
		}
		return res;
	}
	ll fastpow(ll a, ll n, ll p)   //针对特别大的数的快速幂
	{
		ll res = 1;
		a %= p;
		while (n > 0)
		{
			if (n & 1) res = (__int128) res * a % p;
			a = (__int128) a * a % p;
			n >>= 1;
		}
		return res;
	}
	int prime[N], ptot;
	bool ispr[N];
	struct pt
	{
		ll p;
		int c;
	};
	void getprime()   //获取10^6以内的质数
	{
		memset(ispr, 1, sizeof(ispr));
		for (int i = 2; i < N; ++i)
		{
			if (ispr[i]) prime[++ptot] = i;
			for (int j = 1; j <= ptot && prime[j] <= (N - 1) / i; ++j)
			{
				ispr[i * prime[j]] = 0;
				if (!i % prime[j]) break;
			}
		}
	}
	bool cmp(pt x, pt y)
	{
		return x.p < y.p;
	}
	void findorg(vector<pt> &v, ll num)   //num分解质因数
	{
		while (num >= N)     //大于10^6的部分,每次用Pho-Rho算法找出最大的一个质因子,然后除掉即可
		{
			ll maxf = PhoRho::divide(num);
			int cnt = 0;
			while (num % maxf == 0)
			{
				cnt++;
				num = num / maxf;
			}
			v.push_back((pt)
			{
				maxf, cnt
			});
		}
		if (ptot == 0) getprime();
		for (int i = 1; i <= ptot && prime[i] <= num; ++i)   //剩下的就是不大于10^6的质因子了,直接暴力枚举
		{
			if (num % prime[i] == 0)
			{
				int cnt = 0;
				while (num % prime[i] == 0)
				{
					cnt++;
					num /= prime[i];
				}
				v.push_back((pt)
				{
					prime[i], cnt
				});
			}
		}
		if (num > 1) v.push_back((pt)
		{
			num, 1
		});
		sort(v.begin(), v.end(), cmp);
	}
	int getorg(ll p, ll phi, vector<pt> &v)   //获取ord
	{
		for (int k = 2;; k++)
		{
			int flag = 1;
			for (int i = 0; i < (int) v.size(); ++i)
			{
				if (fastpow(k, phi / v[i].p, p) == 1)
				{
					flag = 0;
					break;
				}
			}
			if (flag) return k;
		}
	}
	ll BSGS(ll a, ll b, ll p, ll mod)   //BSGS模板
	{
		a %= mod , b %= mod;
		if (b == 1) return 0;
		if (a == 0)
		{
			if (b == 0) return 1;
			else return -1;
		}
		ll t = 1;
		int m = int(sqrt(1.0 * p) + 1);
		ll base = b;
		unordered_map<ll, ll> vis;
		for (int i = 0; i < m; ++i)
		{
			vis[base] = i;
			base = (__int128) base * a % mod;
		}
		base = fastpow(a, m, mod);
		ll now = t;
		for (int i = 1; i <= m + 1; ++i)
		{
			now = (__int128) now * base % mod;
			if (vis.count(now)) return i * m - vis[now];
		}
		return -1;
	}
	ll getksi(ll g, ll h, ll p, ll c, ll n, ll mod)    //得到合并后的解集,然后上BSGS
	{
		vector<ll> pi;
		ll tp = 1;
		for (int i = 0; i <= c; ++i)
		{
			pi.push_back(tp);
			tp *= p;
		}
		ll gq = fastpow(g, pi[c - 1], mod);
		ll inv = 0;
		tp = 1;
		for (int i = c - 1; i >= 0; --i)
		{
			ll tx = tp * BSGS(gq, fastpow((__int128) h * fastpow(g, pi[c] - inv, mod) % mod, pi[i], mod), p, mod);
			inv += tx;
			tp *= p;
		}
		return inv;
	}
	ll exgcd(ll a, ll b, ll &x, ll &y)   //exgcd模板
	{
		if (b == 0)
		{
			x = 1;
			y = 0;
			return a;
		}
		ll d = exgcd(b, a % b, y, x);
		y -= a / b * x;
		return d;
	}
	ll getinv(ll a, ll p)   //扩欧求逆元
	{
		if (a == 0) return 1;
		ll x, y;
		exgcd(a, p, x, y);
		return (x % p + p) % p;
	}
	ll gcd(ll x, ll y)   //gcd
	{
		if (x % y == 0) return y;
		return gcd(y, x % y);
	}
	ll ExgcdSolve(ll a, ll b, ll c, ll &x, ll &y)   //求解exgcd
	{
		ll d;
		if (c % (d = gcd(a, b))) return -1;
		a /= d;
		b /= d;
		c /= d;
		exgcd(a, b, x, y);
		x = (__int128) x * c % b;
		while (x <= 0) x += b;
		return x;
	}
	ll crt(vector<ll> ksi, vector<pt> v)   //crt
	{
		int sz = v.size();
		ll M = 1, ans = 0;
		vector<ll> m;
		for (int i = 0; i < sz; ++i)
		{
			m.push_back(fastpow(v[i].p, v[i].c));
			M *= m[i];
		}
		for (int i = 0; i < sz; ++i)
		{
			ll Mi = M / m[i];
			ans = ((__int128) ans + (__int128) Mi * getinv(Mi, m[i]) * ksi[i]) % M;
		}
		if (ans < 0) ans += M;
		return ans;
	}
	ll getx(ll h, ll g, ll N, ll mod, vector<pt> &v)   //获取解集,然后用crt合并
	{
		vector<ll> ksi;
		for (pt tp:v)
		{
			ll tg = fastpow(g, N / fastpow(tp.p, tp.c), mod);
			ll th = fastpow(h, N / fastpow(tp.p, tp.c), mod);
			ksi.push_back(getksi(tg, th, tp.p, tp.c, N, mod));
		}
		return crt(ksi, v);
	}
	ll solve(ll a, ll b, ll p)   //求解a^x = b(mod p)的最小解
	{
		if (b == 1) return 0;
		ll phiP = p - 1;
		vector<pt> v;
		findorg(v, phiP);
		int rt = getorg(p, phiP, v);
		ll x = getx(a, rt, phiP, p, v);
		ll y = getx(b, rt, phiP, p, v);
		ll aa = 0, bb = 0;
		if (x == 0)
		{
			if (y == 0) return 1;
			else if (y == 1) return 0;
			else return -1;
		}
		else return ExgcdSolve(x, phiP, y, aa, bb);
	}
};
signed main()
{
	ll p , a , b;
	cin >> p >> a >> b;
	ll ans = DLP::solve(a , b , p);
	if(~ans) cout << ans << '\n';
	else cout << "no solution\n";
	return 0;
}
\end{lstlisting}

\end{document}
