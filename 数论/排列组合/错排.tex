\documentclass[E:/GsjzTle/main/main.tex]{subfiles}

\begin{document}

\textbf{问题:}

有 \(n\) 个小球对应顺序放在 \(n\)
个箱子里,现要求重新排列,使得每个小球都不放原来的位置上,求有多少种排列方式?

\textbf{错排:}

\begin{enumerate}
\def\labelenumi{\arabic{enumi}.}
\item
  数学上我们用 \(D_n\) 表示有 \(n\) 个球 \(n\) 个箱子时的方案数。
\item
  自己简单算一下可以得出: \(D_1=0,D_2=1,D_3=2\)
\item
  来看下 \(D_n(n\ge 4)\) 的情况。要推出怎么做,需要分类讨论。
\item
  不妨分成 \(n-1\) 种情况: \(n\) 号球放进了 \(1\) 号箱子, \(n\)
  号球放进了 \(2\) 号箱子\ldots{} \(n\) 号球放进了 \(n-1\) 号箱子(
  \(n\) 号球不能放进 \(n\) 号箱子)。
\end{enumerate}

注意,分类讨论时要搞清楚是否涵盖了所有的情况。我们可以把所有情况列出来:

\begin{quote}
把 \(n\) 号球放进:

\begin{itemize}
\item
  \(1\) 号箱子
\item
  \(2\) 号箱子
\item
  \ldots\ldots{}
\item
  \((k-1)\) 号箱子
\item
  \(k\) 号箱子
\item
  \((k+1)\) 号箱子
\item
  \ldots\ldots{}
\item
  \((n-1)\) 号箱子
\end{itemize}
\end{quote}

现在,我们只着眼于一种情况: \(n\) 号球放进了 \(k\) 号箱子。

那么此时对于 \(k\) 号小球,就有两种选择:

\begin{enumerate}
\def\labelenumi{\arabic{enumi}.}
\item
  放入 \(n\) 号箱子
\item
  不放入 \(n\) 号箱子
\end{enumerate}

\begin{itemize}
\item
  当\(k\)号小球放入了\(n\)号箱子,剩下的\(n-2\)个小球进行错排,那么答案就是\(D_{n-2}\)
\item
  \(k\)号小球不放入\(n\)号箱子,可以理解为:\\
  剩余\(n-1\)个小球,\(1\)号小球不能放入\(1\)号箱子,\(2\)号小球不能放入\(2\)号箱子,......
  ,
  \(k\)号小球不能放入\(n\)号箱子,......,\(n - 1\)号小球不能放入\(n-1\)号箱子。然后进行错排,那么答案就是\(D_{n-1}\)
\end{itemize}

\textbf{所以递推公式: \(D_{n} = (n - 1)\times(D_{n-1} + D_{n-2})\)
(乘上 \((n -1)\) 是因为 \(k\) 有 \(n-1\) 种选择)}

\textbf{通项公式:\(D_{n}=n!\left[{\frac {1}{2!}}-{\frac {1}{3!}}+...+(-1)^{n}{\frac {1}{n!}}\right]\)}

\end{document}
