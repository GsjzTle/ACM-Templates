\documentclass[E:/GsjzTle/main/main.tex]{subfiles}
\begin{document}

\begin{itemize}
\item
  求最小的 \(x\) 使得 \(a^x\equiv b(mod\;p)\)
\item
  \(p\) 为质数且 \(2\le a,b < p<2^{31}\)
\item
  复杂度为 \(O(\sqrt{p}×log)\) ,\(log\) 为 \(map\) 的复杂度 (卡常可改为
  \(unordered\_map\))
\end{itemize}

\begin{lstlisting}
struct BSGS
{
	ll power(ll a, ll b, ll c) //快速幂
	{
		if(b == 0) return 1 % c;
		ll ans = 1, t = a;
		while (b > 0)
		{
			if (b % 2 == 1) ans = ans * t % c;
			b >>= 1;
			t = t * t % c;
		}
		return ans;
	}
	ll bsgs(ll a, ll b, ll p) //bsgs
	{
		map<ll, ll> hash;
		hash.clear(); //建立一个Hash表
		b %= p;
		ll t = sqrt(p) + 1;
		for(ll i = 0; i < t; ++i) hash[(ll)b * power(a, i, p) % p] = i; //将每个j对应的值插入Hash表
		a = power(a, t, p);
		if(!a) return b == 0 ? 1 : -1; //特判
		for(ll i = 1; i <= t; ++i) //在Hash表中查找是否有i对应的j值
		{
			ll val = power(a, i, p);
			int j = hash.find(val) == hash.end() ? -1 : hash[val];
			if (j >= 0 && i * t - j >= 0)
			return i * t - j;
		}
		return -1; //无解返回-1
	}
} bs;
signed main()
{
	ll p , a , b;
	cin >> p >> a >> b;
	ll ans = bs.bsgs(a, b, p);
	if(~ans) cout << ans << '\n';
	else cout << "no solution\n";
	return 0;
}
\end{lstlisting}

\end{document}
