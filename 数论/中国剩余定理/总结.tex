\documentclass[E:/GsjzTle/main/main.tex]{subfiles}

\begin{document}

中国剩余定理是一种用于求解诸如\\
\(\begin{cases}x≡q_1\;\;(mod\;\;p_1)\\x≡q_2\;\;(mod\;\;p_2)\\ \cdots \cdots\\x≡q_k\;\;(mod\;\;p_k)\\\end{cases}\)\\
形式的同余方程组的定理\\
其中,\(p_1,p_2,...,p_k\) 为\textbf{两两互质}的整数,我们的目的,是找出
\(x\) 的\textbf{最小非负整数}解

\begin{quote}
若求的是大于某个数 (假设这个数是 \(d\)) 的最小整数解\\
则可定义 \(LCM = p1\times p2\times ... \times pn\)\\
那么当 \(x <= d\) 时,让 \(x += LCM\),\(x\%=LCA\) 即可\\
不过 \(\bmod\) 完后 \(x\) 的值可能为 \(0\) , 若为 \(0\) 则 \(LCA\)
为答案
\end{quote}

相比中国剩余定理,扩展中国剩余定理可无视 \(p_1,p_2,...,p_k\)
\textbf{两两互质}的条件

\end{document}
